\documentclass{sprawozdanie}
\usepackage{color}
\usepackage{tabularray}
\usepackage{tikz}
\usepackage{polski}
\usepackage{graphicx} % Required for inserting images
\usepackage{multicol}
\usepackage{booktabs}
\usepackage{cleveref}
\usepackage{subfig}
\usepackage{lipsum}
\usepackage[T1]{fontenc}
% package to open file containing variables
\usepackage{datatool, filecontents}

\usepackage[
backend=biber,
style = verbose
]{biblatex}

\addbibresource{ml_proj10.bib}

\DTLsetseparator{,}% Set the separator between the columns.

% import data
\DTLloaddb[noheader, keys={thekey,thevalue}]{mydata}{../../database/plane_properties.csv}
% Loads mydata.dat with column headers 'thekey' and 'thevalue'
\newcommand{\PlaneVar}[1]{\DTLfetch{mydata}{thekey}{#1}{thevalue}}

\renewcommand\figurename{Rys.}%skrocony podpis
\renewcommand\tablename{Tab.}


\crefname{equation}{Równ.}{Równ.}
\Crefname{equation}{Równ.}{Równ.}
\crefname{figure}{Rys.}{Rys.}
\Crefname{figure}{Rys.}{Rys.}
\crefname{table}{Tab.}{Tab.}
\Crefname{table}{Tab.}{Tab.}


\setlength{\headheight}{14.49998pt}
\addtolength{\topmargin}{-2.49998pt}

\title{„Podłużna statyczna stateczność i sterowność samolotu”}
\author{Marek Polewski}
\class{Mechanika Lotu 2}
\instructor{dr inż. Maciej Lasek}
\projectName{Projekt 10}
\facultyName{Wydział Mechaniczny Energetyki i Lotnictwa Politechniki Warszawskiej - Zakład Mechaniki}



\begin{document}


\maketitle

\leavevmode\thispagestyle{empty}
\newpage


\tableofcontents

\FloatBarrier
\setcounter{page}{1}

\section{Wstęp}

Celem projektu jest sprawdzenie statycznej stateczności i sterowności podłużnej. Obliczenia zostały wykonane w język Python.

Podsatawowe zmienne wykorzystane do obliczeń:
\begin{itemize}
    \item powierzchnia nośna $S = \PlaneVar{S} m^2$,   
    \item powierzchnia steru wysokości $S_H = \PlaneVar{S_h} m^2$,
    \item Średnia cięciwa aerodynamiczna $C_a = \PlaneVar{C_a} m $,
    \item wydłużenie płata $\Lambda = \PlaneVar{Lambda}$,
    \item średnia cięciwa steru wysokości $C_{sH} = \PlaneVar{śr.c_h} m$,
\end{itemize}
Ponaddto zostało założone przełożenie kątowe drążka $K_{dh} = \PlaneVar{ksd}$
\begin{table}[H]
    \centering
    \begin{tabular}{lr} 
    \toprule
                & \multicolumn{1}{l}{$\bar x_c [-] $}  \\  \midrule
    $\bar x_{c_1}$ & 0.12                                \\ 
    $\bar x_{c_2}$ & 0.25                                \\ 
    $\bar x_{c_3}$ & 0.38                                \\ 
    \bottomrule
\end{tabular}
    \label{tab:srodki}
    \caption{Położenia środka ciężkości samolotu}
\end{table}

\section{Środki statecznej statecznosci i sterowności podłużnej}

\begin{align}
    \bar x_N &= \left(\bar x_{SA} + \sum_j \Delta \bar x_{SA_j} + \bar z_s \left(2 \cdot C_z \left(\frac{1}{\pi \Lambda} - \frac{1}{a}\right) - \alpha_0\right) + \kappa'^0_H \cdot\frac{a_1}{a}\cdot\left(1 - \frac{\partial \varepsilon}{\partial \alpha}\right)\right) \cdot K_{gHN} \label{eq:xN}\\
    \bar x_{N'} &= \left(\bar x_{SA} + \sum_j \Delta \bar x_{SA_j} + \bar z_s \left(2 \cdot C_z \left(\frac{1}{\pi \Lambda} - \frac{1}{a}\right) - \alpha_0\right) + \kappa'^0_H \cdot\frac{a_1}{a}\cdot\left(1 - \frac{\partial \varepsilon}{\partial \alpha}\right)\cdot\left(1 - \frac{a_2}{a_1}\cdot\frac{b_1}{b_2}\right)\right) \cdot K_{gHN'} \label{eq:xNprim}\\
    \bar x_M &= \left(\bar x_{SA} + \sum_j \Delta \bar x_{SA_j} + \bar z_s \left(2 \cdot C_z \left(\frac{1}{\pi \Lambda} - \frac{1}{a}\right) - \alpha_0\right) + \kappa'^0_H \cdot\frac{a_1}{a}\cdot\left(\left(1 - \frac{\partial \varepsilon}{\partial \alpha}\right) + \frac{a}{\mu_1^0}\right)\right) \cdot K_{gHM} \label{eq:xM}\\
    \bar x_{M'} &= \left(\bar x_{SA} + \sum_j \Delta \bar x_{SA_j} + \bar z_s \left(2 \cdot C_z \left(\frac{1}{\pi \Lambda} - \frac{1}{a}\right) - \alpha_0\right) + \kappa'^0_H \cdot\frac{a_1}{a}\cdot\left(1 - \frac{a_2}{a_1}\cdot\frac{b_1}{b_2}\right)\cdot\left(\left(1 - \frac{\partial \varepsilon}{\partial \alpha}\right) + \frac{a}{\mu_1^0}\right)\right) \cdot K_{gHM'} \label{eq:xMprim}
\end{align}



% przy czym wielkości \(K_{gHN}\), \(K_{gHN'}\), \(K_{gHM}\), \(K_{gHM'}\) są współczynnikami obliczonymi odpowiednio z Równań \eqref{eq:KgHN}, \eqref{eq:KgHNprim}, \eqref{eq:KgHM} i \eqref{eq:KgHMprim}.

\begin{align}
    K_{gHN} &= \frac{1}{1+\frac{S_H}{S}\cdot\frac{a_1}{a}\cdot\left(\frac{V_{H\infty}}{V_\infty}\right)^2 \cdot \left(1 - \frac{\partial\varepsilon}{\partial\alpha}\right)} = \PlaneVar{kghn} \label{eq:KgHN}\\
    K_{gHN'} &= \frac{1}{1+\frac{S_H}{S}\cdot\frac{a_1}{a}\cdot\left(\frac{V_{H\infty}}{V_\infty}\right)^2 \cdot \left(1 - \frac{\partial\varepsilon}{\partial\alpha}\right)\cdot\left(1 - \frac{a_2}{a_1}\cdot\frac{b_1}{b_2}\right)} = \PlaneVar{kghn'} \label{eq:KgHNprim}\\
    K_{gHN} &= \frac{1}{1+\frac{S_H}{S}\cdot\frac{a_1}{a}\cdot\left(\frac{V_{H\infty}}{V_\infty}\right)^2 \cdot \left(\left(1 - \frac{\partial\varepsilon}{\partial\alpha}\right) + \frac{2 a}{\mu^0_1}\right)}\label{eq:KgHM} = \PlaneVar{kghm}\\
    K_{gHM'} &= \frac{1}{1+\frac{S_H}{S}\cdot\frac{a_1}{a}\cdot\left(\frac{V_{H\infty}}{V_\infty}\right)^2 \cdot \left(\left(1 - \frac{\partial\varepsilon}{\partial\alpha}\right) + \frac{2 a}{\mu_1^0}\right)\cdot\left(1 - \frac{a_2}{a_1}\cdot\frac{b_1}{b_2}\right)} = \PlaneVar{kghm'} \label{eq:KgHMprim}
\end{align}

\begin{equation}
    \mu_1^0 = \frac{m}{\frac{1}{2}\cdot\rho S x_{SAH}} = \PlaneVar{mu}
    \label{eq:muzeroone}
\end{equation}


\begin{equation}
    \kappa'^0_H = \frac{S_H x_{SAH}}{S \cdot c_a}\cdot\left(\frac{V_{H\infty}}{V_\infty}\right) = \PlaneVar{kappa'_h}
    \label{eq:kappazerH}
\end{equation}

\begin{table}[H]
    \let\center\empty
    % \let\endcenter\relax
    \centering
    \resizebox{0.99\width}{!}{\begin{tabular}{rrrrrr}
\toprule
 $V$ [m/s] &  $C_z$ &  $x_n$ &  $x_n'$ &  $x_m$ &  $x_m'$ \\
\midrule
    20.000 &  1.805 &  0.526 &   0.494 &  0.587 &   0.532 \\
    23.684 &  1.287 &  0.485 &   0.452 &  0.546 &   0.490 \\
    27.368 &  0.964 &  0.459 &   0.426 &  0.521 &   0.464 \\
    31.053 &  0.749 &  0.442 &   0.408 &  0.504 &   0.447 \\
    34.737 &  0.598 &  0.430 &   0.396 &  0.492 &   0.435 \\
    38.421 &  0.489 &  0.421 &   0.387 &  0.483 &   0.426 \\
    42.105 &  0.407 &  0.415 &   0.380 &  0.477 &   0.420 \\
    45.789 &  0.344 &  0.410 &   0.375 &  0.472 &   0.415 \\
    49.474 &  0.295 &  0.406 &   0.371 &  0.468 &   0.411 \\
    53.158 &  0.255 &  0.403 &   0.368 &  0.465 &   0.407 \\
    56.842 &  0.223 &  0.400 &   0.366 &  0.462 &   0.405 \\
    60.526 &  0.197 &  0.398 &   0.363 &  0.460 &   0.403 \\
    64.211 &  0.175 &  0.396 &   0.362 &  0.458 &   0.401 \\
    67.895 &  0.157 &  0.395 &   0.360 &  0.457 &   0.399 \\
    71.579 &  0.141 &  0.394 &   0.359 &  0.455 &   0.398 \\
    75.263 &  0.127 &  0.393 &   0.358 &  0.454 &   0.397 \\
    78.947 &  0.116 &  0.392 &   0.357 &  0.454 &   0.396 \\
    82.632 &  0.106 &  0.391 &   0.356 &  0.453 &   0.395 \\
    86.316 &  0.097 &  0.390 &   0.355 &  0.452 &   0.395 \\
    90.000 &  0.089 &  0.390 &   0.355 &  0.451 &   0.394 \\
\bottomrule
\end{tabular}
}
    \caption{Tablica z danymi do obliczeń.}
    \label{tab:tab1}
\end{table}

% insert images from imgs folder with captions and labels xn_cz.png and xn_v.png images side by side with subplots


\begin{figure}[H]
    \centering
    \includegraphics[width=0.8\textwidth]{imgs/xn_cz.png}
    \caption{Wykresy w funkcji $Cz$}
    \label{fig:xn_cz}
\end{figure}

\begin{figure}[H]
    \centering
    \includegraphics[width=0.8\textwidth]{imgs/xn_v.png}
    \caption{Wykresy w funkcji $V$}
    \label{fig:xn_cz2}
\end{figure}

\section{Zapasy podłużnej stateczności i sterowności samolotu}

Kryteria sterownoci podłunej względem prędkoci definiowane jako pochodne kąta
wychylenia steru wysokoci oraz siły na drążku (wolancie) względem prędkoci lotu dane są następującymi zależnościami:

\begin{itemize}
    \item Zapas stateczności ze sterem trzymanym:
    \begin{equation*}
        \bar{h_N} = \bar{x_N}-\bar{x_c} 
    \end{equation*}
    \item Zapas stateczności ze sterem puszczonym:
    \begin{equation*}
        \bar{h_N} = \bar{x_N}-\bar{x_c} 
    \end{equation*}
    \item Zapas sterowności ze sterem trzymanym:
    \begin{equation*}
        \bar{h_N} = \bar{x_N}-\bar{x_c} 
    \end{equation*}
    \item Zapas sterowności ze sterem puszczonym:
    \begin{equation*}
        \bar{h_N} = \bar{x_N}-\bar{x_c} 
    \end{equation*}
\end{itemize}

\begin{figure}[H]
    \centering
    \subfloat[Zapas stateczności ze sterem trzymanym]{\includegraphics[width=0.49\textwidth]{imgs/hn_v1.png}\label{fig:hn_v1}}
    \hfill
    \subfloat[Zapas stateczności ze sterem puszczonym]{\includegraphics[width=0.49\textwidth]{imgs/hn_v2.png}\label{fig:hn_v2}}
    \vfill
    \subfloat[Zapas sterowności ze sterem trzymanym]{\includegraphics[width=0.49\textwidth]{imgs/hn_v3.png}\label{fig:hn_v3}}
    \hfill
    \subfloat[Zapas sterowności ze sterem puszczonym]{\includegraphics[width=0.49\textwidth]{imgs/hn_v4.png}\label{fig:hn_v4}}
    \caption{Zapas stateczności i sterowności w funkcji prędkości}
\end{figure}

\section{Kryterium sterowności podłużnej samolotu}

\subsection{Względem prędkości lotu}
Kryteria sterowności podłunej względem prędkoci definiowane jako pochodne kąta wychylenia steru wysokoci oraz siły na drążku (wolancie) względem prędkoci. Wyniki zostały przedstawione w 

\begin{align}
    \frac{\delta_H}{dV} = \frac{4mg}{\rho S \kappa_H'^{0} \cdot a_2} \cdot \frac{1}{V^3} \cdot \bar h_N \label{ddeltadV} 
\end{align}
\begin{align*}   
    \frac{\delta_H}{dV} = \frac{4\cdot \PlaneVar{m_calc} \cdot 9.81}{1.225 \cdot \PlaneVar{S} \cdot \PlaneVar{kappa'_h}\cdot \PlaneVar{a2}} \cdot \frac{1}{V^3} \cdot \bar h_N  
\end{align*}

\begin{align}
    \frac{dP_{dH}}{dV} = -2mgK_{dH} \cdot \frac{S_{Sh}\cdot  c_{sH}}{S_H \cdot x_{SAH}} \cdot \frac{c_a}{l_{dH}} \cdot \frac{1}{V} \cdot \bar h_{N'}
\end{align}
\begin{align*}   
    \frac{dP_{dH}}{dV} = -2\cdot \PlaneVar{m_calc} \cdot 9.81 \cdot \PlaneVar{kdh} \cdot \frac{\PlaneVar{Sh} \cdot \PlaneVar{śr.c_h}}{\PlaneVar{S} \cdot \PlaneVar{x_sah}} \cdot \frac{\PlaneVar{ca}}{\PlaneVar{ldH}} \cdot \frac{1}{V} \cdot \bar h_{N'}
\end{align*}


\begin{figure}[H]
    \centering
    \subfloat[Pochodna wychylenia stery wysokości]{\includegraphics[width=0.8\textwidth]{imgs/d_delta_v.png}\label{fig:deltah_v}}
    \vfill
    \subfloat[Przyrost siły na drążku]{\includegraphics[width=0.8\textwidth]{imgs/d_P_v.png}\label{fig:dPdV}}

    \caption{Kryteria sterowności podłużnej w funkcji prędkości}
\end{figure}

\subsection{Względem przeciążenia}

Kryteria sterowności podłużnej względem przeciążenia: przyrost kąta wychylenia względem współczynnika przeciążeń i przyrost siły na drążku odniesiony do współczynnika przeciążeń

\begin{equation}
    \frac{\Delta\delta_H}{m_g-1} = - 2 \cdot m \cdot g \cdot \frac{c_a}{\rho \cdot S_H \cdot x_{SAH}\cdot a_2}\cdot \frac{1}{V^2}\cdot \bar h_M
    \label{eq:DdHmg1}
\end{equation}

\begin{equation*}
    \frac{\Delta\delta_H}{m_g-1} = - 2 \cdot \PlaneVar{m_calc} \cdot 9.81 \cdot \frac{\PlaneVar{ca}}{1.225 \cdot \PlaneVar{Sh} \cdot \PlaneVar{x_sah}\cdot \PlaneVar{a2}}\cdot \frac{1}{V^2}\cdot \bar h_M
\end{equation*}

\begin{equation}
    \frac{\Delta P_{dH}}{m_g-1} = m\cdot g \cdot K_{dH} \cdot \frac{c_a}{l_{dH}} \cdot \frac{S_{SH} \cdot c_{SH}}{S_H \cdot x_{SAH} \cdot \frac{b_2}{a_2}}\cdot \bar h_{M'}
    \label{eq:DPdHmg}
\end{equation}

\begin{equation*}
    \frac{\Delta P_{dH}}{m_g-1} = \PlaneVar{m_calc} \cdot 9.81 \cdot \PlaneVar{kdh} \cdot \frac{\PlaneVar{ca}}{\PlaneVar{ldH}} \cdot \frac{\PlaneVar{Sh} \cdot \PlaneVar{śr.c_h}}{\PlaneVar{S} \cdot \PlaneVar{x_sah} \cdot \frac{\PlaneVar{b2}}{\PlaneVar{a2}}}\cdot \bar h_{M'}
\end{equation*}

\begin{figure}[H]
    \centering
    \subfloat[Pochodna wychylenia stery wysokości]{\includegraphics[width=0.8\textwidth]{imgs/delta_delta_v.png}\label{fig:deltah_mg}}
    \vfill
    \subfloat[Przyrost siły na drążku]{\includegraphics[width=0.8\textwidth]{imgs/delta_p_v.png}\label{fig:dPdmg}}

    \caption{Kryteria sterowności podłużnej w funkcji przeciążenia}
\end{figure}

\subsection{Podsumowanie}

Na podsatwie warunków zawartych w poniżej można swierdzić, że samolot jest sterowny podłużnie.
\begin{align*}
    \frac{d\delta_H}{dV}>0 \quad \frac{dP_{dH}}{dV}<0 \quad \frac{\Delta\delta_H}{m_g-1}<0 \quad \frac{\Delta P_{dH}}{m_g-1}>0
\end{align*}


\end{document}