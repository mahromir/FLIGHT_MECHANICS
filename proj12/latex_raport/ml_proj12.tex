\documentclass{sprawozdanie}
\usepackage{color}
\usepackage{tabularray}
\usepackage{tikz}
\usepackage{polski}
\usepackage{graphicx} % Required for inserting images
\usepackage{multicol}
\usepackage{booktabs}
\usepackage{cleveref}

\usepackage{lipsum}
\usepackage[T1]{fontenc}
% package to open file containing variables
\usepackage{datatool, filecontents}

\usepackage[
backend=biber,
style = verbose
]{biblatex}

\addbibresource{ml_proj9.bib}


\DTLsetseparator{,}% Set the separator between the columns.

% import data
\DTLloaddb[noheader, keys={thekey,thevalue}]{mydata}{../../database/plane_properties.csv}
% Loads mydata.dat with column headers 'thekey' and 'thevalue'
\newcommand{\PlaneVar}[1]{\DTLfetch{mydata}{thekey}{#1}{thevalue}}

\renewcommand\figurename{Rys.}%skrocony podpis
\renewcommand\tablename{Tab.}


\crefname{equation}{Równ.}{Równ.}
\Crefname{equation}{Równ.}{Równ.}
\crefname{figure}{Rys.}{Rys.}
\Crefname{figure}{Rys.}{Rys.}
\crefname{table}{Tab.}{Tab.}
\Crefname{table}{Tab.}{Tab.}


\setlength{\headheight}{14.49998pt}
\addtolength{\topmargin}{-2.49998pt}

\title{„Analiza ruchów fugoidalnych samolotu”}
\author{Marek Polewski}
\class{Mechanika Lotu 2}
\instructor{dr. inż Maciej Lasek}
\projectName{Projekt 12}
\facultyName{Wydział Mechaniczny Energetyki i Lotnictwa Politechniki Warszawskiej - Zakład Mechaniki}



\begin{document}

\maketitle

\newpage


\setcounter{page}{1}



\section{Wstęp}
Obliczenia zostały wykonane w programie EXCEL. Wyszystkie obliczenia zostały wkononane dla samolotu o następujących parametrach:
\begin{itemize}
    \item $m = 700kg$,
    \item $S = 15 m^2$,
    \item $h = 3km$,
    \item $\rho(h = 3km) = 0.908 kgm^{-3}$
\end{itemize}

\section{Równania ruchu samolotu}
Zastosowane zostały poniższe równania ruchu samolotu:
\begin{align}
    m \frac{dv}{dt} + mg sin(\vartheta) + \frac{1}{2}\rho S(V_0+v)^2 C_x = 0
\end{align}
\begin{align}
    mV_0 \frac{d\vartheta}{dt} + mg cos(\vartheta) - \frac{1}{2}\rho S(V_0+v)^2 C_z = 0
\end{align}
Równania te zostały sprawdzone do bezwymiarowych równań ruchu:
\begin{align}
    \frac{d\bar v}{d\bar t} + \frac{1}{2} C_z \vartheta + C_x \bar v = 0
\end{align}
\begin{align}
    \frac{d\bar \vartheta}{d\bar t} + \frac{1}{2} C_x \vartheta + C_z \bar v = 0
\end{align}
gdzie $\bar v = \frac{v}{V_0}$, $\bar \vartheta $ - bezwymiarowe zaburzenie kąta pochylenia, $\bar t$ bezwymiarowy czas, $\hat t = \frac{2m}{\rho S V_0}$ - czas aerodynamiczny

Pierwiaskami zespolonymi sprzeżnomi równań (1) i (2) są:
\begin{align*}
    \lambda_{1,2} = -\frac{3}{4} C_x \pm i \cdot \sqrt{\frac{1}{2} \Big( C_x^2+C_z^2 \Big) - \frac{9}{16} C_x^2}
\end{align*}
Wówczas otrzymyjemy:\\
Bezwymiarowy współczynnik tłuamienia:
\begin{align*}
    \bar \zeta = -\frac{3}{4} C_x
\end{align*}
Bezwymiarowa częstość ruchu okresowego:
\begin{align*}
    \bar \eta = \sqrt{\frac{1}{2} \Big( C_x^2+C_z^2 \Big) - \frac{9}{16} C_x^2}
\end{align*}
Okres ruchu:
\begin{align*}
    T = \frac{w \pi}{\sqrt{\frac{1}{2} \Big( C_x^2+C_z^2 \Big) - \frac{9}{16} C_x^2}} \cdot i
\end{align*}
czas stłumienia amplotudy do $\frac{1}{2}$:
\begin{align*}
    T_{\frac{1}{2}} = \frac{\ln 2}{\frac{3}{4}C_x}\cdot i
\end{align*}
Otrzymane wartości zamieszczam w tabeli:
% insert table table.tex here
\begin{table}[H]
    \centering
    \begin{tabular}{rrrrrrrr}
   \toprule
    $C_z$ &  $C_x$ &  $V [m/s]$ &  $Im$ &  $Re$ &  $\hat t$ &  $T [s]$ &  $T_{\frac{1}{2}}$ [s] \\
   \midrule
     0.10 &   0.05 &   100.37 &  0.07 & -0.04 &      1.02 &    92.40 &                  18.70 \\
     0.17 &   0.05 &    77.39 &  0.12 & -0.04 &      1.33 &    70.52 &                  23.76 \\
     0.24 &   0.05 &    65.28 &  0.17 & -0.04 &      1.57 &    59.32 &                  27.34 \\
     0.30 &   0.06 &    57.51 &  0.21 & -0.04 &      1.79 &    52.20 &                  29.85 \\
     0.37 &   0.06 &    51.99 &  0.26 & -0.04 &      1.98 &    47.16 &                  31.53 \\
     0.44 &   0.06 &    47.80 &  0.31 & -0.05 &      2.15 &    43.35 &                  32.52 \\
     0.51 &   0.06 &    44.48 &  0.36 & -0.05 &      2.31 &    40.33 &                  32.96 \\
     0.58 &   0.07 &    41.77 &  0.41 & -0.05 &      2.46 &    37.87 &                  32.95 \\
     0.65 &   0.07 &    39.51 &  0.46 & -0.06 &      2.60 &    35.81 &                  32.60 \\
     0.71 &   0.08 &    37.57 &  0.50 & -0.06 &      2.73 &    34.06 &                  32.00 \\
     0.78 &   0.08 &    35.90 &  0.55 & -0.06 &      2.86 &    32.54 &                  31.20 \\
     0.85 &   0.09 &    34.43 &  0.60 & -0.07 &      2.98 &    31.20 &                  30.27 \\
     0.92 &   0.10 &    33.12 &  0.65 & -0.07 &      3.10 &    30.02 &                  29.26 \\
     0.99 &   0.11 &    31.96 &  0.70 & -0.08 &      3.21 &    28.97 &                  28.20 \\
     1.05 &   0.11 &    30.91 &  0.75 & -0.08 &      3.32 &    28.02 &                  27.12 \\
     1.12 &   0.12 &    29.95 &  0.79 & -0.09 &      3.43 &    27.15 &                  26.04 \\
     1.19 &   0.13 &    29.08 &  0.84 & -0.10 &      3.53 &    26.36 &                  24.98 \\
     1.26 &   0.14 &    28.29 &  0.89 & -0.11 &      3.63 &    25.64 &                  23.95 \\
     1.33 &   0.15 &    27.55 &  0.94 & -0.11 &      3.73 &    24.97 &                  22.94 \\
     1.40 &   0.16 &    26.87 &  0.99 & -0.12 &      3.82 &    24.36 &                  21.98 \\
     1.46 &   0.17 &    26.23 &  1.03 & -0.13 &      3.91 &    23.78 &                  21.06 \\
     1.53 &   0.18 &    25.64 &  1.08 & -0.14 &      4.00 &    23.25 &                  20.18 \\
     1.60 &   0.20 &    25.09 &  1.13 & -0.15 &      4.09 &    22.75 &                  19.35 \\
   \bottomrule
   \end{tabular}
   
    \caption{Wartości parametrów ruchu samolotu}
\end{table}

\begin{figure}[H]
    \centering
    \includegraphics[width=0.6\textwidth]{imgs/reim.png}
    \caption{Bezwymiarowe wartości w czasie}
    \label{fig:fig1}
\end{figure}

\begin{figure}[H]
    \centering
    \includegraphics[width=0.6\textwidth]{imgs/T.png}
    \caption{Parametry okresowe ruchów fugoidalnych}
\end{figure}

\section{Ruchy fugoidalne}
Do obliczeń wybrałem prędkość $V = 44.5 ms^{-1}$, oraz zakłucenie prędkości $v_0 = 0.5 ms^{-1}$.
\begin{align*}
    -\frac{2(\bar \xi +C_x)}{C_z} \pm i \frac{2\bar \eta}{C_z} = a \pm i b = -\frac{2*(-0.0485)*0.0647}{0.5091} \pm i \frac{2\cdot 0.3596}{0.5091} \approx 0.1905 \pm i 1.4142
\end{align*}
Ubezwymiarowany współczynnik prędkości samolotu:
\begin{align*}
    \bar u_o = \frac{v_0}{V_0} = \frac{0.5}{44.5} = 0.0112
\end{align*}

Ruchy fugoidalne samolotu mają postać:
\begin{align*}
    \vartheta_0 (t) = \vartheta_{0} + \vartheta (\bar t) = \frac{\bar u_0}{b} (a^2+b^2) \cdot e^{\bar \xi \frac{t}{\hat t}} \sin(\bar \eta \frac{t}{\hat t} )
\end{align*}
\begin{align*}
    V(t) = V_0 + \bar \bar V(t)\cdot V_0 = V_0 + V_0 \cdot \bar u_0 e^{\bar \xi \frac{t}{\hat t}} \Big( \cos \Big( \bar \eta \frac{t}{\hat t} \Big)   - \frac{a}{b} \sin \Big( \bar \eta \frac{t}{\hat t} \Big) \Big)
\end{align*}

\begin{figure}[H]
    \centering
    \includegraphics[width=0.8\textwidth]{imgs/fugo.png}
    \caption{Ruchy fugoidalne samolotu}

\end{figure}

\section{Wnioski}
Dla prędkości $V = 44.5 ms^{-1}$ oraz zakłócenia prędkości $v_0 = 0.5 ms^{-1}$, samolot \textbf{wykonał ruchy fugoidalne tłumione} z okresem $T \approx 35s$ i czasem stłumienia $T_{\frac{1}{2}} \approx 100s$. Po 200 sekundach samolot wykonuje ruchy o amplitudzie znikomo małej.  

\end{document}