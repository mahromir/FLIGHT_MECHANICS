\documentclass{sprawozdanie}
\usepackage{color}
\usepackage{tabularray}
\usepackage{tikz}
\usepackage{polski}
\usepackage{graphicx} % Required for inserting images
\usepackage{multicol}
\usepackage{booktabs}
\usepackage{cleveref}
\usepackage{subfig}
\usepackage{lipsum}
\usepackage[T1]{fontenc}
% package to open file containing variables
\usepackage{datatool, filecontents}

\usepackage[
backend=biber,
style = verbose
]{biblatex}

\addbibresource{ml_proj10.bib}

\DTLsetseparator{,}% Set the separator between the columns.

% import data
\DTLloaddb[noheader, keys={thekey,thevalue}]{mydata}{../../database/plane_properties.csv}
% Loads mydata.dat with column headers 'thekey' and 'thevalue'
\newcommand{\PlaneVar}[1]{\DTLfetch{mydata}{thekey}{#1}{thevalue}}

\renewcommand\figurename{Rys.}%skrocony podpis
\renewcommand\tablename{Tab.}


\crefname{equation}{Równ.}{Równ.}
\Crefname{equation}{Równ.}{Równ.}
\crefname{figure}{Rys.}{Rys.}
\Crefname{figure}{Rys.}{Rys.}
\crefname{table}{Tab.}{Tab.}
\Crefname{table}{Tab.}{Tab.}


\setlength{\headheight}{14.49998pt}
\addtolength{\topmargin}{-2.49998pt}

\title{„Analiza zakrętu samolotu”}
\author{Marek Polewski}
\class{Mechanika Lotu 2}
\instructor{dr inż. Maciej Lasek}
\projectName{Projekt 11}
\facultyName{Wydział Mechaniczny Energetyki i Lotnictwa Politechniki Warszawskiej - Zakład Mechaniki}



\begin{document}


\maketitle




\FloatBarrier
\setcounter{page}{1}

\section{Wstęp - Naliza parametrów zakrętu}

\begin{align*}
    R = \frac{V^2}{g\cdot tg(\phi)} \quad c_z = \frac{2mg}{\rho S V^2} \cdot \frac{1}{cos (\phi)} \quad m_g = \frac{1}{cos(\phi)} \quad N_{n_zark} = \frac{N_n}{(\sqrt{cos(\phi)})^3}
\end{align*}

Dane wykorzystne do obliczeń:
\begin{itemize}
    \item Masa samolotu podczas lotu $m = 650 kg$
    \item Współczynnik obciążenia $m_g = 2.5$
    \item Wysokość lotu $h = 0 m.n.p.m$ na które gęstość powietrza $\rho = 1.225 \frac{kg}{m^3}$
    \item Dopuszczalne przechylenie $\varphi_{max} = \arccos{\frac{1}{m_g}} = \arccos{\frac{1}{2.5}} = 1.159279 \; rad = 66.42^{\circ} $
\end{itemize}
\section{Promień zakrętu}

% paste table from R_table.tex file

\begin{table}[H]
    \centering
    \resizebox{0.99\width}{!}{\begin{tabular}{lrrrrr}
\toprule
{} &  $\varphi = 15^\circ$ &  $\varphi = 30^\circ$ &  $\varphi = 45^\circ$ &  $\varphi = 60^\circ$ &  $\varphi = 66.6^\circ$ \\
\midrule
v &                 26.81 &                 28.31 &                 31.33 &                 37.26 &                   41.80 \\
r &                273.35 &                141.50 &                100.05 &                 81.69 &                   77.09 \\
\bottomrule
\end{tabular}
}
    % \begin{tabular}{rrrrr}
\toprule
   R\_15 &   R\_30 &   R\_45 &   R\_60 &  R\_66.6 \\
\midrule
   0.00 &   0.00 &   0.00 &   0.00 &    0.00 \\
   1.52 &   0.71 &   0.41 &   0.24 &    0.18 \\
   6.09 &   2.82 &   1.63 &   0.94 &    0.71 \\
  13.70 &   6.36 &   3.67 &   2.12 &    1.59 \\
  24.35 &  11.30 &   6.52 &   3.77 &    2.82 \\
  38.04 &  17.66 &  10.19 &   5.89 &    4.41 \\
  54.78 &  25.42 &  14.68 &   8.47 &    6.35 \\
  74.56 &  34.61 &  19.98 &  11.54 &    8.65 \\
  97.39 &  45.20 &  26.10 &  15.07 &   11.29 \\
 123.26 &  57.21 &  33.03 &  19.07 &   14.29 \\
 152.17 &  70.62 &  40.77 &  23.54 &   17.64 \\
 184.13 &  85.45 &  49.34 &  28.48 &   21.35 \\
 219.13 & 101.70 &  58.72 &  33.90 &   25.41 \\
 257.17 & 119.35 &  68.91 &  39.78 &   29.82 \\
 298.26 & 138.42 &  79.92 &  46.14 &   34.58 \\
 342.39 & 158.90 &  91.74 &  52.97 &   39.70 \\
 389.56 & 180.80 & 104.38 &  60.27 &   45.17 \\
 439.78 & 204.10 & 117.84 &  68.03 &   50.99 \\
 493.04 & 228.82 & 132.11 &  76.27 &   57.17 \\
 549.35 & 254.95 & 147.20 &  84.98 &   63.70 \\
 608.69 & 282.50 & 163.10 &  94.17 &   70.58 \\
 671.08 & 311.45 & 179.82 & 103.82 &   77.81 \\
 736.52 & 341.82 & 197.35 & 113.94 &   85.40 \\
 805.00 & 373.60 & 215.70 & 124.53 &   93.34 \\
 876.52 & 406.79 & 234.86 & 135.60 &  101.63 \\
 951.08 & 441.40 & 254.84 & 147.13 &  110.28 \\
1028.69 & 477.42 & 275.64 & 159.14 &  119.28 \\
1109.34 & 514.85 & 297.25 & 171.62 &  128.63 \\
1193.04 & 553.69 & 319.67 & 184.56 &  138.34 \\
1279.78 & 593.95 & 342.92 & 197.98 &  148.39 \\
1369.56 & 635.61 & 366.97 & 211.87 &  158.80 \\
1462.39 & 678.70 & 391.85 & 226.23 &  169.57 \\
1558.25 & 723.19 & 417.53 & 241.06 &  180.68 \\
1657.17 & 769.09 & 444.04 & 256.36 &  192.15 \\
1759.12 & 816.41 & 471.36 & 272.14 &  203.97 \\
1864.12 & 865.14 & 499.49 & 288.38 &  216.15 \\
\bottomrule
\end{tabular}

    \caption{Promień zakrętu}
    \label{tab:R}
\end{table}

\begin{figure}[H]
    \centering
    \includegraphics[width=0.8\textwidth]{img/R_plot.png}
    \caption{Promień zakrętu}
    \label{fig:R}
\end{figure}

Jak widać na wykresie \ref{fig:R} promień zakrętu jest najmniejszy dla dużych kątów przechylenia. Osiąga on stosunkowo niewielkie wartości, co wskazuje na dobrą manewrowość. Cecha ta jest zazwyzcaj porządana dla samolotu szkolno-treningowego. Dane zaprezentowane na wykresie pochodzą z \cref{tab:Rbig}.

\begin{table}[H]
    \centering
    \resizebox{0.9\width}{!}{\begin{tabular}{rrrrr}
\toprule
   R\_15 &   R\_30 &   R\_45 &   R\_60 &  R\_66.6 \\
\midrule
   0.00 &   0.00 &   0.00 &   0.00 &    0.00 \\
   1.52 &   0.71 &   0.41 &   0.24 &    0.18 \\
   6.09 &   2.82 &   1.63 &   0.94 &    0.71 \\
  13.70 &   6.36 &   3.67 &   2.12 &    1.59 \\
  24.35 &  11.30 &   6.52 &   3.77 &    2.82 \\
  38.04 &  17.66 &  10.19 &   5.89 &    4.41 \\
  54.78 &  25.42 &  14.68 &   8.47 &    6.35 \\
  74.56 &  34.61 &  19.98 &  11.54 &    8.65 \\
  97.39 &  45.20 &  26.10 &  15.07 &   11.29 \\
 123.26 &  57.21 &  33.03 &  19.07 &   14.29 \\
 152.17 &  70.62 &  40.77 &  23.54 &   17.64 \\
 184.13 &  85.45 &  49.34 &  28.48 &   21.35 \\
 219.13 & 101.70 &  58.72 &  33.90 &   25.41 \\
 257.17 & 119.35 &  68.91 &  39.78 &   29.82 \\
 298.26 & 138.42 &  79.92 &  46.14 &   34.58 \\
 342.39 & 158.90 &  91.74 &  52.97 &   39.70 \\
 389.56 & 180.80 & 104.38 &  60.27 &   45.17 \\
 439.78 & 204.10 & 117.84 &  68.03 &   50.99 \\
 493.04 & 228.82 & 132.11 &  76.27 &   57.17 \\
 549.35 & 254.95 & 147.20 &  84.98 &   63.70 \\
 608.69 & 282.50 & 163.10 &  94.17 &   70.58 \\
 671.08 & 311.45 & 179.82 & 103.82 &   77.81 \\
 736.52 & 341.82 & 197.35 & 113.94 &   85.40 \\
 805.00 & 373.60 & 215.70 & 124.53 &   93.34 \\
 876.52 & 406.79 & 234.86 & 135.60 &  101.63 \\
 951.08 & 441.40 & 254.84 & 147.13 &  110.28 \\
1028.69 & 477.42 & 275.64 & 159.14 &  119.28 \\
1109.34 & 514.85 & 297.25 & 171.62 &  128.63 \\
1193.04 & 553.69 & 319.67 & 184.56 &  138.34 \\
1279.78 & 593.95 & 342.92 & 197.98 &  148.39 \\
1369.56 & 635.61 & 366.97 & 211.87 &  158.80 \\
1462.39 & 678.70 & 391.85 & 226.23 &  169.57 \\
1558.25 & 723.19 & 417.53 & 241.06 &  180.68 \\
1657.17 & 769.09 & 444.04 & 256.36 &  192.15 \\
1759.12 & 816.41 & 471.36 & 272.14 &  203.97 \\
1864.12 & 865.14 & 499.49 & 288.38 &  216.15 \\
\bottomrule
\end{tabular}
}
    % \begin{tabular}{rrrrr}
\toprule
   R\_15 &   R\_30 &   R\_45 &   R\_60 &  R\_66.6 \\
\midrule
   0.00 &   0.00 &   0.00 &   0.00 &    0.00 \\
   1.52 &   0.71 &   0.41 &   0.24 &    0.18 \\
   6.09 &   2.82 &   1.63 &   0.94 &    0.71 \\
  13.70 &   6.36 &   3.67 &   2.12 &    1.59 \\
  24.35 &  11.30 &   6.52 &   3.77 &    2.82 \\
  38.04 &  17.66 &  10.19 &   5.89 &    4.41 \\
  54.78 &  25.42 &  14.68 &   8.47 &    6.35 \\
  74.56 &  34.61 &  19.98 &  11.54 &    8.65 \\
  97.39 &  45.20 &  26.10 &  15.07 &   11.29 \\
 123.26 &  57.21 &  33.03 &  19.07 &   14.29 \\
 152.17 &  70.62 &  40.77 &  23.54 &   17.64 \\
 184.13 &  85.45 &  49.34 &  28.48 &   21.35 \\
 219.13 & 101.70 &  58.72 &  33.90 &   25.41 \\
 257.17 & 119.35 &  68.91 &  39.78 &   29.82 \\
 298.26 & 138.42 &  79.92 &  46.14 &   34.58 \\
 342.39 & 158.90 &  91.74 &  52.97 &   39.70 \\
 389.56 & 180.80 & 104.38 &  60.27 &   45.17 \\
 439.78 & 204.10 & 117.84 &  68.03 &   50.99 \\
 493.04 & 228.82 & 132.11 &  76.27 &   57.17 \\
 549.35 & 254.95 & 147.20 &  84.98 &   63.70 \\
 608.69 & 282.50 & 163.10 &  94.17 &   70.58 \\
 671.08 & 311.45 & 179.82 & 103.82 &   77.81 \\
 736.52 & 341.82 & 197.35 & 113.94 &   85.40 \\
 805.00 & 373.60 & 215.70 & 124.53 &   93.34 \\
 876.52 & 406.79 & 234.86 & 135.60 &  101.63 \\
 951.08 & 441.40 & 254.84 & 147.13 &  110.28 \\
1028.69 & 477.42 & 275.64 & 159.14 &  119.28 \\
1109.34 & 514.85 & 297.25 & 171.62 &  128.63 \\
1193.04 & 553.69 & 319.67 & 184.56 &  138.34 \\
1279.78 & 593.95 & 342.92 & 197.98 &  148.39 \\
1369.56 & 635.61 & 366.97 & 211.87 &  158.80 \\
1462.39 & 678.70 & 391.85 & 226.23 &  169.57 \\
1558.25 & 723.19 & 417.53 & 241.06 &  180.68 \\
1657.17 & 769.09 & 444.04 & 256.36 &  192.15 \\
1759.12 & 816.41 & 471.36 & 272.14 &  203.97 \\
1864.12 & 865.14 & 499.49 & 288.38 &  216.15 \\
\bottomrule
\end{tabular}

    \caption{Promień zakrętu}
    \label{tab:Rbig}
\end{table}




\section{Współczynnik siły nośnej }

\begin{table}[H]
    \centering
    \begin{tabular}{lrrrrr}
\toprule
{} &  $C_z (\varphi = 15^\circ)$ &  $C_z (\varphi = 30^\circ)$ &  $C_z (\varphi = 45^\circ)$ &  $C_z (\varphi = 60^\circ)$ &  $C_z (\varphi = 66.6^\circ)$ \\
\midrule
0.0  &                         inf &                         inf &                         inf &                         inf &                           inf \\
2.0  &                      179.63 &                      200.35 &                      245.38 &                      347.02 &                        436.89 \\
4.0  &                       44.91 &                       50.09 &                       61.35 &                       86.76 &                        109.22 \\
6.0  &                       19.96 &                       22.26 &                       27.26 &                       38.56 &                         48.54 \\
8.0  &                       11.23 &                       12.52 &                       15.34 &                       21.69 &                         27.31 \\
10.0 &                        7.19 &                        8.01 &                        9.82 &                       13.88 &                         17.48 \\
12.0 &                        4.99 &                        5.57 &                        6.82 &                        9.64 &                         12.14 \\
14.0 &                        3.67 &                        4.09 &                        5.01 &                        7.08 &                          8.92 \\
16.0 &                        2.81 &                        3.13 &                        3.83 &                        5.42 &                          6.83 \\
18.0 &                        2.22 &                        2.47 &                        3.03 &                        4.28 &                          5.39 \\
20.0 &                        1.80 &                        2.00 &                        2.45 &                        3.47 &                          4.37 \\
22.0 &                        1.48 &                        1.66 &                        2.03 &                        2.87 &                          3.61 \\
24.0 &                        1.25 &                        1.39 &                        1.70 &                        2.41 &                          3.03 \\
26.0 &                        1.06 &                        1.19 &                        1.45 &                        2.05 &                          2.59 \\
28.0 &                        0.92 &                        1.02 &                        1.25 &                        1.77 &                          2.23 \\
30.0 &                        0.80 &                        0.89 &                        1.09 &                        1.54 &                          1.94 \\
32.0 &                        0.70 &                        0.78 &                        0.96 &                        1.36 &                          1.71 \\
34.0 &                        0.62 &                        0.69 &                        0.85 &                        1.20 &                          1.51 \\
36.0 &                        0.55 &                        0.62 &                        0.76 &                        1.07 &                          1.35 \\
38.0 &                        0.50 &                        0.55 &                        0.68 &                        0.96 &                          1.21 \\
40.0 &                        0.45 &                        0.50 &                        0.61 &                        0.87 &                          1.09 \\
42.0 &                        0.41 &                        0.45 &                        0.56 &                        0.79 &                          0.99 \\
44.0 &                        0.37 &                        0.41 &                        0.51 &                        0.72 &                          0.90 \\
46.0 &                        0.34 &                        0.38 &                        0.46 &                        0.66 &                          0.83 \\
48.0 &                        0.31 &                        0.35 &                        0.43 &                        0.60 &                          0.76 \\
50.0 &                        0.29 &                        0.32 &                        0.39 &                        0.56 &                          0.70 \\
52.0 &                        0.27 &                        0.30 &                        0.36 &                        0.51 &                          0.65 \\
54.0 &                        0.25 &                        0.27 &                        0.34 &                        0.48 &                          0.60 \\
56.0 &                        0.23 &                        0.26 &                        0.31 &                        0.44 &                          0.56 \\
58.0 &                        0.21 &                        0.24 &                        0.29 &                        0.41 &                          0.52 \\
60.0 &                        0.20 &                        0.22 &                        0.27 &                        0.39 &                          0.49 \\
62.0 &                        0.19 &                        0.21 &                        0.26 &                        0.36 &                          0.45 \\
64.0 &                        0.18 &                        0.20 &                        0.24 &                        0.34 &                          0.43 \\
66.0 &                        0.16 &                        0.18 &                        0.23 &                        0.32 &                          0.40 \\
68.0 &                        0.16 &                        0.17 &                        0.21 &                        0.30 &                          0.38 \\
70.0 &                        0.15 &                        0.16 &                        0.20 &                        0.28 &                          0.36 \\
72.0 &                        0.14 &                        0.15 &                        0.19 &                        0.27 &                          0.34 \\
74.0 &                        0.13 &                        0.15 &                        0.18 &                        0.25 &                          0.32 \\
76.0 &                        0.12 &                        0.14 &                        0.17 &                        0.24 &                          0.30 \\
78.0 &                        0.12 &                        0.13 &                        0.16 &                        0.23 &                          0.29 \\
80.0 &                        0.11 &                        0.13 &                        0.15 &                        0.22 &                          0.27 \\
\bottomrule
\end{tabular}

    \caption{Współczynnik siły nośnej}
    \label{tab:c_z}
\end{table}

\begin{figure}[H]
    \centering
    \includegraphics[width=0.8\textwidth]{img/cz_plot.png}
    \caption{Współczynnik siły nośnej}
    \label{fig:c_z}
\end{figure}


\section{Moc}

\begin{sidewaystable}
    \centering
    \begin{tabular}{lrrrrrrr}
\toprule
{} &  $P_n (\varphi = 15^\circ)$ [kW] &  $P_n (\varphi = 30^\circ)$ [kW] &  $P_n (\varphi = 45^\circ)$ [kW] &  $P_n (\varphi = 60^\circ)$ [kW] &  $P_n (\varphi = 66.6^\circ)$ [kW] &  $P_n(\varphi = {0}^\circ)$ [kW] &  $N_r$ [kW] \\
\midrule
6.0  &                      15503544.96 &                      17291957.50 &                      21178236.27 &                      29950548.96 &                        37707047.76 &                      14975274.48 &        0.03 \\
8.0  &                        637065.33 &                        710554.05 &                        870247.43 &                       1230715.71 &                         1549442.59 &                        615357.86 &        0.03 \\
10.0 &                         52025.15 &                         58026.51 &                         71067.67 &                        100504.87 &                          126533.30 &                         50252.43 &        0.04 \\
12.0 &                          6440.71 &                          7183.68 &                          8798.17 &                         12442.49 &                           15664.81 &                          6221.25 &        0.04 \\
14.0 &                          1049.72 &                          1170.81 &                          1433.94 &                          2027.90 &                            2553.08 &                          1013.95 &        0.04 \\
16.0 &                           216.45 &                           241.42 &                           295.67 &                           418.15 &                             526.44 &                           209.07 &        0.04 \\
18.0 &                            64.10 &                            71.49 &                            87.56 &                           123.83 &                             155.89 &                            61.91 &        0.05 \\
20.0 &                            33.69 &                            37.58 &                            46.02 &                            65.09 &                              81.94 &                            32.54 &        0.05 \\
22.0 &                            27.67 &                            30.86 &                            37.80 &                            53.45 &                              67.30 &                            26.73 &        0.05 \\
24.0 &                            26.82 &                            29.91 &                            36.64 &                            51.81 &                              65.23 &                            25.90 &        0.06 \\
26.0 &                            27.16 &                            30.30 &                            37.11 &                            52.48 &                              66.07 &                            26.24 &        0.06 \\
28.0 &                            27.93 &                            31.16 &                            38.16 &                            53.96 &                              67.94 &                            26.98 &        0.06 \\
30.0 &                            29.04 &                            32.39 &                            39.67 &                            56.10 &                              70.63 &                            28.05 &        0.06 \\
32.0 &                            30.51 &                            34.03 &                            41.68 &                            58.95 &                              74.22 &                            29.48 &        0.06 \\
34.0 &                            32.40 &                            36.14 &                            44.26 &                            62.60 &                              78.81 &                            31.30 &        0.06 \\
36.0 &                            34.74 &                            38.74 &                            47.45 &                            67.10 &                              84.48 &                            33.55 &        0.07 \\
38.0 &                            37.54 &                            41.87 &                            51.28 &                            72.52 &                              91.30 &                            36.26 &        0.07 \\
40.0 &                            40.83 &                            45.54 &                            55.77 &                            78.87 &                              99.30 &                            39.44 &        0.07 \\
42.0 &                            44.62 &                            49.77 &                            60.95 &                            86.20 &                             108.53 &                            43.10 &        0.07 \\
44.0 &                            48.93 &                            54.58 &                            66.85 &                            94.54 &                             119.02 &                            47.27 &        0.07 \\
46.0 &                            53.78 &                            59.99 &                            73.47 &                           103.90 &                             130.81 &                            51.95 &        0.07 \\
48.0 &                            59.18 &                            66.01 &                            80.84 &                           114.32 &                             143.93 &                            57.16 &        0.07 \\
50.0 &                            65.14 &                            72.66 &                            88.99 &                           125.84 &                             158.43 &                            62.92 &        0.08 \\
52.0 &                            71.69 &                            79.96 &                            97.93 &                           138.49 &                             174.35 &                            69.24 &        0.08 \\
54.0 &                            78.83 &                            87.93 &                           107.69 &                           152.29 &                             191.73 &                            76.15 &        0.08 \\
56.0 &                            86.60 &                            96.59 &                           118.29 &                           167.29 &                             210.62 &                            83.65 &        0.08 \\
58.0 &                            95.00 &                           105.95 &                           129.77 &                           183.52 &                             231.05 &                            91.76 &        0.08 \\
60.0 &                           104.05 &                           116.05 &                           142.14 &                           201.01 &                             253.07 &                           100.51 &        0.08 \\
62.0 &                           113.78 &                           126.91 &                           155.43 &                           219.81 &                             276.73 &                           109.90 &        0.08 \\
64.0 &                           124.20 &                           138.53 &                           169.67 &                           239.94 &                             302.08 &                           119.97 &        0.08 \\
66.0 &                           135.34 &                           150.95 &                           184.88 &                           261.46 &                             329.17 &                           130.73 &        0.09 \\
68.0 &                           147.21 &                           164.20 &                           201.10 &                           284.39 &                             358.05 &                           142.20 &        0.09 \\
70.0 &                           159.84 &                           178.28 &                           218.34 &                           308.79 &                             388.75 &                           154.39 &        0.09 \\
72.0 &                           173.24 &                           193.22 &                           236.65 &                           334.67 &                             421.34 &                           167.34 &        0.09 \\
74.0 &                           187.43 &                           209.06 &                           256.04 &                           362.10 &                             455.87 &                           181.05 &        0.09 \\
76.0 &                           202.45 &                           225.80 &                           276.55 &                           391.10 &                             492.38 &                           195.55 &        0.09 \\
78.0 &                           218.29 &                           243.48 &                           298.19 &                           421.71 &                             530.92 &                           210.86 &        0.09 \\
80.0 &                           235.00 &                           262.11 &                           321.01 &                           453.98 &                             571.55 &                           226.99 &        0.09 \\
\bottomrule
\end{tabular}

    \caption{Niezbędna moc rozporządzalna do lotu}
    \label{tab:Pbig}
\end{sidewaystable}

\begin{figure}[H]
    \centering
    \includegraphics[width=0.8\textwidth]{img/Pr_plot.png}
    \caption{Moc}
    \label{fig:P}
\end{figure}

Jak widać na wykresie powyżej, nie wszystkie kąty przechylenia są osiągalne przy danej mocy silnika. W celu polepszenia parametrów zakrętu należy zwiększyć moc silnika. Dane zaprezentowane na wykresie pochodzą z \ref{tab:Pbig}.

\newpage
\section{Zakręt ze stałym kątem natarcia}
Do wyznaczenia wartości zakrętu na stałym kącie natarcia został przyjęty  $C_z = 1.53*0.8 = 1.225$.
\begin{figure}[H]
    \centering
    \includegraphics[width=0.8\textwidth]{img/mg_plot.png}
    \caption{Współczynnik obciążenia $m_g$ w zakręcie na stałym kącie natarcia}
    \label{fig:mg}
\end{figure}
Na ostatnim wykresie znajdują się informacje o promieniu zakrętu oraz przechyleniu w zakręcie przy utrzymaniu stałego kąta natarcia. Dodatkowo zaznaczone są ograniczenia wynikające z maksymalnego współczynnika siły nośnej, ciągu silnika, prędkości maksymalnej i minimalnej, a także z maksymalnego współczynnika obciążenia.

\begin{figure}[H]
    \centering
    \includegraphics[width=0.8\textwidth]{img/phi_r_plot.png}
    \caption{Zakręt przy stałym kącie natarcia}
    \label{fig:N}
\end{figure}



\end{document}