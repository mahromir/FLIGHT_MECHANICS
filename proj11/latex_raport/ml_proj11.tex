\documentclass{sprawozdanie}
\usepackage{color}
\usepackage{tabularray}
\usepackage{tikz}
\usepackage{polski}
\usepackage{graphicx} % Required for inserting images
\usepackage{multicol}
\usepackage{booktabs}
\usepackage{cleveref}
\usepackage{subfig}
\usepackage{lipsum}
\usepackage[T1]{fontenc}
% package to open file containing variables
\usepackage{datatool, filecontents}

\usepackage[
backend=biber,
style = verbose
]{biblatex}

\addbibresource{ml_proj10.bib}

\DTLsetseparator{,}% Set the separator between the columns.

% import data
\DTLloaddb[noheader, keys={thekey,thevalue}]{mydata}{../../database/plane_properties.csv}
% Loads mydata.dat with column headers 'thekey' and 'thevalue'
\newcommand{\PlaneVar}[1]{\DTLfetch{mydata}{thekey}{#1}{thevalue}}

\renewcommand\figurename{Rys.}%skrocony podpis
\renewcommand\tablename{Tab.}


\crefname{equation}{Równ.}{Równ.}
\Crefname{equation}{Równ.}{Równ.}
\crefname{figure}{Rys.}{Rys.}
\Crefname{figure}{Rys.}{Rys.}
\crefname{table}{Tab.}{Tab.}
\Crefname{table}{Tab.}{Tab.}


\setlength{\headheight}{14.49998pt}
\addtolength{\topmargin}{-2.49998pt}

\title{„Podłużna statyczna stateczność i sterowność samolotu”}
\author{Marek Polewski}
\class{Mechanika Lotu 2}
\instructor{dr inż. Maciej Lasek}
\projectName{Projekt 10}
\facultyName{Wydział Mechaniczny Energetyki i Lotnictwa Politechniki Warszawskiej - Zakład Mechaniki}



\begin{document}


\maketitle

\leavevmode\thispagestyle{empty}


\tableofcontents

\FloatBarrier
\setcounter{page}{1}

\section{Wstęp - Naliza parametrów zakrętu}

\begin{align*}
    R = \frac{V^2}{g\cdot tg(\phi)} \quad c_z = \frac{2mg}{\rho S V^2} \cdot \frac{1}{cos (\phi)} \quad m_g = \frac{1}{cos(\phi)} \quad N_{n_zark} = \frac{N_n}{(\sqrt{cos(\phi)})^3}
\end{align*}

\section{Promień zakrętu}

% paste table from R_table.tex file
\begin{table}[H]
    \centering
    \begin{tabular}{lrrrrr}
\toprule
{} &  $R(\varphi = 15^\circ)$ &  $R(\varphi = 30^\circ)$ &  $R(\varphi = 45^\circ)$ &  $R(\varphi = 60^\circ)$ &  $R(\varphi = 66.6^\circ)$ \\
\midrule
0.0  &                     0.00 &                     0.00 &                     0.00 &                     0.00 &                       0.00 \\
2.0  &                     1.52 &                     0.71 &                     0.41 &                     0.24 &                       0.18 \\
4.0  &                     6.09 &                     2.82 &                     1.63 &                     0.94 &                       0.71 \\
6.0  &                    13.70 &                     6.36 &                     3.67 &                     2.12 &                       1.59 \\
8.0  &                    24.35 &                    11.30 &                     6.52 &                     3.77 &                       2.82 \\
10.0 &                    38.04 &                    17.66 &                    10.19 &                     5.89 &                       4.41 \\
12.0 &                    54.78 &                    25.42 &                    14.68 &                     8.47 &                       6.35 \\
14.0 &                    74.56 &                    34.61 &                    19.98 &                    11.54 &                       8.65 \\
16.0 &                    97.39 &                    45.20 &                    26.10 &                    15.07 &                      11.29 \\
18.0 &                   123.26 &                    57.21 &                    33.03 &                    19.07 &                      14.29 \\
20.0 &                   152.17 &                    70.62 &                    40.77 &                    23.54 &                      17.64 \\
22.0 &                   184.13 &                    85.45 &                    49.34 &                    28.48 &                      21.35 \\
24.0 &                   219.13 &                   101.70 &                    58.72 &                    33.90 &                      25.41 \\
26.0 &                   257.17 &                   119.35 &                    68.91 &                    39.78 &                      29.82 \\
28.0 &                   298.26 &                   138.42 &                    79.92 &                    46.14 &                      34.58 \\
30.0 &                   342.39 &                   158.90 &                    91.74 &                    52.97 &                      39.70 \\
32.0 &                   389.56 &                   180.80 &                   104.38 &                    60.27 &                      45.17 \\
34.0 &                   439.78 &                   204.10 &                   117.84 &                    68.03 &                      50.99 \\
36.0 &                   493.04 &                   228.82 &                   132.11 &                    76.27 &                      57.17 \\
38.0 &                   549.35 &                   254.95 &                   147.20 &                    84.98 &                      63.70 \\
40.0 &                   608.69 &                   282.50 &                   163.10 &                    94.17 &                      70.58 \\
42.0 &                   671.08 &                   311.45 &                   179.82 &                   103.82 &                      77.81 \\
44.0 &                   736.52 &                   341.82 &                   197.35 &                   113.94 &                      85.40 \\
46.0 &                   805.00 &                   373.60 &                   215.70 &                   124.53 &                      93.34 \\
48.0 &                   876.52 &                   406.79 &                   234.86 &                   135.60 &                     101.63 \\
50.0 &                   951.08 &                   441.40 &                   254.84 &                   147.13 &                     110.28 \\
52.0 &                  1028.69 &                   477.42 &                   275.64 &                   159.14 &                     119.28 \\
54.0 &                  1109.34 &                   514.85 &                   297.25 &                   171.62 &                     128.63 \\
56.0 &                  1193.04 &                   553.69 &                   319.67 &                   184.56 &                     138.34 \\
58.0 &                  1279.78 &                   593.95 &                   342.92 &                   197.98 &                     148.39 \\
60.0 &                  1369.56 &                   635.61 &                   366.97 &                   211.87 &                     158.80 \\
62.0 &                  1462.39 &                   678.70 &                   391.85 &                   226.23 &                     169.57 \\
64.0 &                  1558.25 &                   723.19 &                   417.53 &                   241.06 &                     180.68 \\
66.0 &                  1657.17 &                   769.09 &                   444.04 &                   256.36 &                     192.15 \\
68.0 &                  1759.12 &                   816.41 &                   471.36 &                   272.14 &                     203.97 \\
70.0 &                  1864.12 &                   865.14 &                   499.49 &                   288.38 &                     216.15 \\
72.0 &                  1972.17 &                   915.29 &                   528.44 &                   305.10 &                     228.68 \\
74.0 &                  2083.25 &                   966.84 &                   558.21 &                   322.28 &                     241.56 \\
76.0 &                  2197.38 &                  1019.81 &                   588.79 &                   339.94 &                     254.79 \\
78.0 &                  2314.56 &                  1074.19 &                   620.18 &                   358.06 &                     268.38 \\
80.0 &                  2434.77 &                  1129.98 &                   652.40 &                   376.66 &                     282.32 \\
\bottomrule
\end{tabular}

    \caption{Promień zakrętu}
    \label{tab:R}
\end{table}

\begin{figure}[H]
    \centering
    \includegraphics[width=0.8\textwidth]{img/R_plot.png}
    \caption{Promień zakrętu}
    \label{fig:R}
\end{figure}

\section{Współczynnik siły nośnej }

\begin{table}[H]
    \centering
    \begin{tabular}{rrrrr}
\toprule
 cz\_15 &  cz\_30 &  cz\_45 &  cz\_60 &  cz\_66.6 \\
\midrule
   inf &    inf &    inf &    inf &      inf \\
179.63 & 200.35 & 245.38 & 347.02 &   436.89 \\
 44.91 &  50.09 &  61.35 &  86.76 &   109.22 \\
 19.96 &  22.26 &  27.26 &  38.56 &    48.54 \\
 11.23 &  12.52 &  15.34 &  21.69 &    27.31 \\
  7.19 &   8.01 &   9.82 &  13.88 &    17.48 \\
  4.99 &   5.57 &   6.82 &   9.64 &    12.14 \\
  3.67 &   4.09 &   5.01 &   7.08 &     8.92 \\
  2.81 &   3.13 &   3.83 &   5.42 &     6.83 \\
  2.22 &   2.47 &   3.03 &   4.28 &     5.39 \\
  1.80 &   2.00 &   2.45 &   3.47 &     4.37 \\
  1.48 &   1.66 &   2.03 &   2.87 &     3.61 \\
  1.25 &   1.39 &   1.70 &   2.41 &     3.03 \\
  1.06 &   1.19 &   1.45 &   2.05 &     2.59 \\
  0.92 &   1.02 &   1.25 &   1.77 &     2.23 \\
  0.80 &   0.89 &   1.09 &   1.54 &     1.94 \\
  0.70 &   0.78 &   0.96 &   1.36 &     1.71 \\
  0.62 &   0.69 &   0.85 &   1.20 &     1.51 \\
  0.55 &   0.62 &   0.76 &   1.07 &     1.35 \\
  0.50 &   0.55 &   0.68 &   0.96 &     1.21 \\
  0.45 &   0.50 &   0.61 &   0.87 &     1.09 \\
  0.41 &   0.45 &   0.56 &   0.79 &     0.99 \\
  0.37 &   0.41 &   0.51 &   0.72 &     0.90 \\
  0.34 &   0.38 &   0.46 &   0.66 &     0.83 \\
  0.31 &   0.35 &   0.43 &   0.60 &     0.76 \\
  0.29 &   0.32 &   0.39 &   0.56 &     0.70 \\
  0.27 &   0.30 &   0.36 &   0.51 &     0.65 \\
  0.25 &   0.27 &   0.34 &   0.48 &     0.60 \\
  0.23 &   0.26 &   0.31 &   0.44 &     0.56 \\
  0.21 &   0.24 &   0.29 &   0.41 &     0.52 \\
  0.20 &   0.22 &   0.27 &   0.39 &     0.49 \\
  0.19 &   0.21 &   0.26 &   0.36 &     0.45 \\
  0.18 &   0.20 &   0.24 &   0.34 &     0.43 \\
  0.16 &   0.18 &   0.23 &   0.32 &     0.40 \\
  0.16 &   0.17 &   0.21 &   0.30 &     0.38 \\
  0.15 &   0.16 &   0.20 &   0.28 &     0.36 \\
\bottomrule
\end{tabular}

    \caption{Współczynnik siły nośnej}
    \label{tab:c_z}
\end{table}

\begin{figure}[H]
    \centering
    \includegraphics[width=0.8\textwidth]{img/cz_plot.png}
    \caption{Współczynnik siły nośnej}
    \label{fig:c_z}
\end{figure}

\section{Moc}

\begin{table}[H]
    \centering
    \begin{tabular}{lrrrrrrr}
\toprule
{} &  $P_n (\varphi = 15^\circ)$ [kW] &  $P_n (\varphi = 30^\circ)$ [kW] &  $P_n (\varphi = 45^\circ)$ [kW] &  $P_n (\varphi = 60^\circ)$ [kW] &  $P_n (\varphi = 66.6^\circ)$ [kW] &  $P_n(\varphi = {0}^\circ)$ [kW] &  $N_r$ [kW] \\
\midrule
6.0  &                      15503544.96 &                      17291957.50 &                      21178236.27 &                      29950548.96 &                        37707047.76 &                      14975274.48 &        0.03 \\
8.0  &                        637065.33 &                        710554.05 &                        870247.43 &                       1230715.71 &                         1549442.59 &                        615357.86 &        0.03 \\
10.0 &                         52025.15 &                         58026.51 &                         71067.67 &                        100504.87 &                          126533.30 &                         50252.43 &        0.04 \\
12.0 &                          6440.71 &                          7183.68 &                          8798.17 &                         12442.49 &                           15664.81 &                          6221.25 &        0.04 \\
14.0 &                          1049.72 &                          1170.81 &                          1433.94 &                          2027.90 &                            2553.08 &                          1013.95 &        0.04 \\
16.0 &                           216.45 &                           241.42 &                           295.67 &                           418.15 &                             526.44 &                           209.07 &        0.04 \\
18.0 &                            64.10 &                            71.49 &                            87.56 &                           123.83 &                             155.89 &                            61.91 &        0.05 \\
20.0 &                            33.69 &                            37.58 &                            46.02 &                            65.09 &                              81.94 &                            32.54 &        0.05 \\
22.0 &                            27.67 &                            30.86 &                            37.80 &                            53.45 &                              67.30 &                            26.73 &        0.05 \\
24.0 &                            26.82 &                            29.91 &                            36.64 &                            51.81 &                              65.23 &                            25.90 &        0.06 \\
26.0 &                            27.16 &                            30.30 &                            37.11 &                            52.48 &                              66.07 &                            26.24 &        0.06 \\
28.0 &                            27.93 &                            31.16 &                            38.16 &                            53.96 &                              67.94 &                            26.98 &        0.06 \\
30.0 &                            29.04 &                            32.39 &                            39.67 &                            56.10 &                              70.63 &                            28.05 &        0.06 \\
32.0 &                            30.51 &                            34.03 &                            41.68 &                            58.95 &                              74.22 &                            29.48 &        0.06 \\
34.0 &                            32.40 &                            36.14 &                            44.26 &                            62.60 &                              78.81 &                            31.30 &        0.06 \\
36.0 &                            34.74 &                            38.74 &                            47.45 &                            67.10 &                              84.48 &                            33.55 &        0.07 \\
38.0 &                            37.54 &                            41.87 &                            51.28 &                            72.52 &                              91.30 &                            36.26 &        0.07 \\
40.0 &                            40.83 &                            45.54 &                            55.77 &                            78.87 &                              99.30 &                            39.44 &        0.07 \\
42.0 &                            44.62 &                            49.77 &                            60.95 &                            86.20 &                             108.53 &                            43.10 &        0.07 \\
44.0 &                            48.93 &                            54.58 &                            66.85 &                            94.54 &                             119.02 &                            47.27 &        0.07 \\
46.0 &                            53.78 &                            59.99 &                            73.47 &                           103.90 &                             130.81 &                            51.95 &        0.07 \\
48.0 &                            59.18 &                            66.01 &                            80.84 &                           114.32 &                             143.93 &                            57.16 &        0.07 \\
50.0 &                            65.14 &                            72.66 &                            88.99 &                           125.84 &                             158.43 &                            62.92 &        0.08 \\
52.0 &                            71.69 &                            79.96 &                            97.93 &                           138.49 &                             174.35 &                            69.24 &        0.08 \\
54.0 &                            78.83 &                            87.93 &                           107.69 &                           152.29 &                             191.73 &                            76.15 &        0.08 \\
56.0 &                            86.60 &                            96.59 &                           118.29 &                           167.29 &                             210.62 &                            83.65 &        0.08 \\
58.0 &                            95.00 &                           105.95 &                           129.77 &                           183.52 &                             231.05 &                            91.76 &        0.08 \\
60.0 &                           104.05 &                           116.05 &                           142.14 &                           201.01 &                             253.07 &                           100.51 &        0.08 \\
62.0 &                           113.78 &                           126.91 &                           155.43 &                           219.81 &                             276.73 &                           109.90 &        0.08 \\
64.0 &                           124.20 &                           138.53 &                           169.67 &                           239.94 &                             302.08 &                           119.97 &        0.08 \\
66.0 &                           135.34 &                           150.95 &                           184.88 &                           261.46 &                             329.17 &                           130.73 &        0.09 \\
68.0 &                           147.21 &                           164.20 &                           201.10 &                           284.39 &                             358.05 &                           142.20 &        0.09 \\
70.0 &                           159.84 &                           178.28 &                           218.34 &                           308.79 &                             388.75 &                           154.39 &        0.09 \\
72.0 &                           173.24 &                           193.22 &                           236.65 &                           334.67 &                             421.34 &                           167.34 &        0.09 \\
74.0 &                           187.43 &                           209.06 &                           256.04 &                           362.10 &                             455.87 &                           181.05 &        0.09 \\
76.0 &                           202.45 &                           225.80 &                           276.55 &                           391.10 &                             492.38 &                           195.55 &        0.09 \\
78.0 &                           218.29 &                           243.48 &                           298.19 &                           421.71 &                             530.92 &                           210.86 &        0.09 \\
80.0 &                           235.00 &                           262.11 &                           321.01 &                           453.98 &                             571.55 &                           226.99 &        0.09 \\
\bottomrule
\end{tabular}

    \caption{Moc}
    \label{tab:P}
\end{table}

\begin{figure}[H]
    \centering
    \includegraphics[width=0.8\textwidth]{img/Pr_plot.png}
    \caption{Moc}
    \label{fig:P}
\end{figure}



\printbibliography

\end{document}