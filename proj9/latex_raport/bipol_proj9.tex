\documentclass[12pt]{sprawozdanie}
\usepackage{color}
\usepackage{tabularray}
\usepackage{tikz}
\usepackage{polski}
\usepackage{graphicx} % Required for inserting images
\usepackage{multicol}
\usepackage{booktabs}

\usepackage{lipsum}
\usepackage[T1]{fontenc}
% package to open file containing variables
\usepackage{datatool, filecontents}
\DTLsetseparator{,}% Set the separator between the columns.

% import data
\DTLloaddb[noheader, keys={thekey,thevalue}]{mydata}{../../database/plane_properties.csv}
% Loads mydata.dat with column headers 'thekey' and 'thevalue'
\newcommand{\PlaneVar}[1]{\DTLfetch{mydata}{thekey}{#1}{thevalue}}




\setlength{\headheight}{14.49998pt}
\addtolength{\topmargin}{-2.49998pt}

\title{„Równowaga podłużna samolotu i siły na sterownicy wysokości”}
\author{Marek Polewski}
\class{Mechanika Lotu 2}
\instructor{dr. inż Maciej Lasek}
\projectName{Projekt 9}
\facultyName{Wydział Mechaniczny Energetyki i Lotnictwa Politechniki Warszawskiej - Zakład Mechaniki}

\renewcommand\figurename{Rys.}%skrocony podpis
\renewcommand\tablename{Tab.}


\begin{document}

\maketitle

\leavevmode\thispagestyle{empty}
\newpage


\tableofcontents

\FloatBarrier
\setcounter{page}{1}


% \section{Samochodzik}

% Zamiast wciągać kresli wole głaskać pieski

% \begin{figure}[h!]
%     \centering
%     \includegraphics[width = 0.4\linewidth]{imgs/burek.jpg}
%     \caption{Burek.jpg}
%     \label{fig:burek}
% \end{figure}
% \FloatBarrier


\section{Wstęp}
Celem projektu jest wyznaczenie kątów wychylenia usterzenia poziomego samolotu koniecznych do zachowania równowangi oraz wyznaczenia siły na drążku.
Obliczenia zostały wykonane w Pythonie. \\
Wysokość przyjęta do obliczeń to $h=3000 m$. Gęstość jej odpowiadająca wynosi $\rho = 0.908 kgm^{-3}$.  \\
Dane z poprzednich projektów:
\begin{itemize}
    \item powierzchnia skrzydeł $S = \PlaneVar{S} m^2$,
    \item powierzchnia usterzenia poziomego $S_h = \PlaneVar{S_h} m^2$,
    \item $\Big(\frac{V_{H\infty}}{V_{\infty}}\Big)^2 = 0.85 $,
\end{itemize}


% TODO FIX citation
Do obliczeń zostały wykorzystane wartości środków ciężkości wyliczone \cite{Proj8}


\begin{table}[h!]
    \centering
    \begin{tabular}{lr} 
    \toprule
                & \multicolumn{1}{l}{$\bar x_c [-] $}  \\ \hline
    $\bar x_c_1$ & \PlaneVar{bx_c1}                                \\ \hline
    $\bar x_c_2$ & \PlaneVar{bx_c2}                                \\ \hline
    $\bar x_c_3$ & \PlaneVar{bx_c3}                                \\ 
    \bottomrule
    \end{tabular}
    \label{tab:srodki}
    \caption{Położenia środka ciężkości samolotu}
\end{table}
\FloatBarrier

\section{Współczynnik momentu podłużnego usterzenia wysokości}
\subsection{Cecha objętościowa usterzenia poziomego}

Wzór na cechę usterzenia:

\begin{equation}
    \kappa'_{h} =  (\bar{x}_{SA_H}-\bar{x}_c)\cdot \frac{S_H}{S} \cdot \Big(\frac{V_{H\infty}}{V_{\infty}}\Big)^2
\end{equation}
, gdzie $\bar{x}_{SA_H} =\PlaneVar{bx_sah}$.  \\
Na podstawie powyższego wzoru uzyskałem:

\begin{table}[h!]
    \centering
    
    \begin{tabular}{rr} 
    \toprule
    \multicolumn{1}{l}{$\bar x_{i}$} & \multicolumn{1}{l}{$\kappa'_{Hi}$}  \\ \hline
    \PlaneVar{bx_c1}                             & \PlaneVar{kappa_c1}                               \\ \hline
    \PlaneVar{bx_c2}                             & \PlaneVar{kappa_c2}                               \\ \hline
    \PlaneVar{bx_c3}                             & \PlaneVar{kappa_c3}                               \\
    \bottomrule
    \end{tabular}
    \caption{Cecha objętościowa usterzenia poziomego}
    \label{tab:kappa}
    \end{table}
\FloatBarrier

\subsection{Współczynnik siły nośnej usterzenia poziomego}
Współczynnik siły nośnej usterzenia poziomego wyrażam

\begin{equation*}
    C_{z_H} = a_1 \alpha_H + a_2 \delta_H + a_3 \delta_{H_k}
\end{equation*}


\begin{itemize}
    \item procetowa grubość profilu: $10 \% $,
    \item położenie szczeliny w stosunku do średniej cięciwy płata stabilizatora $0.428\% $,
    \item wydłużenie efektywne usterzenia poziomego: $\Lambda_H = \PlaneVar{Lambda_h}$ 
    \item odwrotność zbierzności usterzenia poziomego: $\frac{1}{\lambda_H} = \PlaneVar{1/lambda_h}$
\end{itemize}

\begin{figure}[h!]
    \centering
    \includegraphics[width = 0.8\linewidth]{imgs/dcz_da_h.jpg}
    \caption{Odczyt charakterystyki $a_1 = \frac{dC_z}{d \alpha}$ dla usterzenia poziomego}
    \label{fig:dczdah}
\end{figure}
\FloatBarrier

Wyznaczanie współcznnika $a_2$

\begin{equation*}
    a_2 = 1.27 \cdot a_1 \cdot \sqrt{\frac{S_{sH}}{S_H}} \Big( 1- 0.2 \cdot \frac{S_{sH}}{S_H} \Big)
\end{equation*}

gdzie $\frac{S_{sH}}{S_H} = \frac{\PlaneVar{SsH}}{\PlaneVar{S_h}} = \PlaneVar{SsH/S_h}$ to stosunek powierzchni steru (\ref{fig:SsH} w \ref{appendix:obl}) do powierzchni usterzenia. 
   
\begin{equation} 
    a_2 = 1.27 \cdot \PlaneVar{a1} \cdot \sqrt{\PlaneVar{SsH/S_h}}\Big( 1- 0.2 \cdot \PlaneVar{SsH/S_h} \Big) = \PlaneVar{a2} rad^{-1}
\end{equation}

\section{Kąt zaklinowania usterzenia wysokości}
Obliczenia przyjmę dla prędkości przelotowej równej $V= 106kt = 54.4 m/s$ oraz wysokości $h = 0 m$. Przy tej prędkości płat będzie musiał wytworzyć współczynnik $C_z$ o pewnej wartości. 

\begin{equation}
    C_z = \frac{2mg}{\rho S V^2} = \frac{2 \cdot \PlaneVar{mtow} \cdot 9.81 }{1.225 \cdot \PlaneVar{S} \cdot \PlaneVar{v_cruise}^2} = 0.262
\end{equation}
    
Równanie kąta zaklinowania łopaty:

\begin{equation*}
    \alpha _{zH} = \frac{C_{mbu}}{\kappa'_{H}\cdot a_1}-\frac{C_z}{a}\cdot \Big( 1- \frac{\partial \varepsilon}{\partial \alpha}  \Big)
\end{equation*}

gdzie

\begin{equation}
    \frac{\partial \varepsilon}{\partial \alpha} = \frac{2a}{\pi \Lambda} = \frac{2 \cdot \PlaneVar{a}}{\pi \cdot \PlaneVar{Lambda}} = \PlaneVar{deps_dalpha}
\end{equation}


% \usepackage{tabularray}
\begin{table}[h!]
    \centering
    \begin{tabular}{rrrrr} 
    \toprule
    \multicolumn{1}{l}{$\bar{x}_c$} & \multicolumn{1}{l}{$C_{mbu}$} & \multicolumn{1}{l}{$\kappa'_H$} & \multicolumn{1}{l}{$\alpha_{zH} [rad]$} & \multicolumn{1}{l}{\begin{tabular}[c]{@{}l@{}}$\alpha_{zH} [deg]$\\\end{tabular}}  \\ 
    \hline
    0.12                            & -0.074                      & 0.34                            & -0.10                                & -5.95                                                                           \\ 
    \hline
    0.25                            & -0.039                     & 0.32                        & -0.07                                & -4.16                                                                           \\ 
    \hline
    0.38                            & -0.004                      & 0.31                        & -0.03                                & -2.17                                                                           \\
    \bottomrule
    \end{tabular}
    \caption{Wyznaczanie $\alpha_{zH}$  - kąta zaklinowania płata}
    \label{tab:azh}
    \end{table}
\FloatBarrier

Jako podstawową wartość przyjmuję $\alpha_{zH} \approx -0.1 rad^{-1}= -5.95 deg$.

\section{Kąt wychylenia steru}
Kąt wychylenia steru $\delta_h$ z równania równowagi:

\begin{equation}
    \delta_h = \frac{C_{mbu}}{\kappa'_{H}\cdot a_2}-\frac{a_1}{a_2}\cdot \Bigg[\frac{C_z}{a} \cdot \Big (1- \frac{\partial \varepsilon}{\partial \alpha} \Big)+\alpha_{zH} \Bigg]
\end{equation}


\begin{figure}[h!]
    \centering
    \includegraphics[width = 0.75\linewidth]{imgs/delta_h_cz.jpg}
    \caption{Kąt wychylenia steru niezbędny do zachowania równowagi dla różnego $C_z$}
    \label{fig:deltahcz}
\end{figure}
\FloatBarrier

\begin{figure}[h!]
    \centering
    \includegraphics[width = 0.75\linewidth]{imgs/delta_h_v.jpg}

    \caption{Kąt wychylenia steru niezbędny do zachowania równowagi dla różnego $V$}
    \label{fig:deltahv}
\end{figure}
\FloatBarrier

\begin{table}[h!]
    \centering
    \begin{tabular}{rrrrr}
        \toprule
        $V [ms^{-1}]$ &   $C_z$ &  $\delta_{h1}[^{\circ}]$ &  $\delta_{h2}[^{\circ}]$ &  $\delta_{h3}[^{\circ}]$ \\
        \midrule
        24  &  1.253129 & -12.720390 & -10.166218 & -7.391849 \\
        25  &  1.154884 & -11.712590 &  -9.158418 & -6.384049 \\
        26  &  1.067755 & -10.818822 &  -8.264650 & -5.490281 \\
        27  &  0.990127 & -10.022511 &  -7.468339 & -4.693970 \\
        28  &  0.920666 &  -9.309986 &  -6.755814 & -3.981445 \\
        29  &  0.858267 &  -8.669892 &  -6.115720 & -3.341351 \\
        30  &  0.802003 &  -8.092734 &  -5.538562 & -2.764193 \\
        31  &  0.751095 &  -7.570524 &  -5.016352 & -2.241983 \\
        32  &  0.704885 &  -7.096503 &  -4.542331 & -1.767962 \\
        33  &  0.662812 &  -6.664917 &  -4.110745 & -1.336376 \\
        34  &  0.624397 &  -6.270850 &  -3.716678 & -0.942309 \\
        35  &  0.589226 &  -5.910076 &  -3.355904 & -0.581535 \\
        36  &  0.556946 &  -5.578946 &  -3.024774 & -0.250405 \\
        37  &  0.527248 &  -5.274300 &  -2.720128 &  0.054241 \\
        38  &  0.499863 &  -4.993387 &  -2.439215 &  0.335154 \\
        39  &  0.474558 &  -4.733804 &  -2.179632 &  0.594737 \\
        40  &  0.451127 &  -4.493446 &  -1.939274 &  0.835095 \\
        41  &  0.429389 &  -4.270460 &  -1.716288 &  1.058081 \\
        42  &  0.409185 &  -4.063210 &  -1.509039 &  1.265331 \\
        43  &  0.390374 &  -3.870252 &  -1.316080 &  1.458289 \\
        44  &  0.372832 &  -3.690299 &  -1.136127 &  1.638242 \\
        45  &  0.356446 &  -3.522210 &  -0.968038 &  1.806331 \\
        46  &  0.341116 &  -3.364963 &  -0.810791 &  1.963578 \\
        47  &  0.326755 &  -3.217646 &  -0.663474 &  2.110895 \\
        48  &  0.313282 &  -3.079440 &  -0.525268 &  2.249101 \\
        49  &  0.300626 &  -2.949609 &  -0.395437 &  2.378932 \\
        50  &  0.288721 &  -2.827490 &  -0.273318 &  2.501051 \\
        51  &  0.277510 &  -2.712483 &  -0.158312 &  2.616058 \\
        52  &  0.266939 &  -2.604048 &  -0.049876 &  2.724493 \\
        53  &  0.256961 &  -2.501692 &   0.052480 &  2.826849 \\
        \bottomrule
        \end{tabular}
    \caption{Kąt wychylenia stery niezbędnego do zachowania równowagi}
    \label{tab:deltahv}        
\end{table}
\FloatBarrier

% \begin{thebibliography}{9}
%     \bibitem{Proj8}
%     Projekt 8 - Moment podłużny samolotu, Marek Polewski październik 2023
% \end{thebibliography}

\appendix
\chapter{Obliczenia powierzchni steru}\label{appendix:obl}
\begin{figure}[h!]
    \centering
    \includegraphics[width = 0.5\linewidth]{imgs/Obliczenia_sh.jpg}
    \caption{Obliczanie pola $S_{sH}$ z wysunku z Projektu 1 \cite{Proj1}}
    \label{fig:SsH}
\end{figure}
\FloatBarrier



\bibliographystyle{plain} % We choose the "plain" reference style
\bibliography{refs} % Entries are in the refs.bib file

\end{document}