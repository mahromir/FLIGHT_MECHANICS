\documentclass[12pt]{sprawozdanie}
\usepackage{color}
\usepackage{tabularray}
\usepackage{tikz}
\usepackage{polski}
\usepackage{graphicx} % Required for inserting images
\usepackage{multicol}
\usepackage{booktabs}

\usepackage{lipsum}
\usepackage[T1]{fontenc}
% package to open file containing variables
\usepackage{datatool, filecontents}
\DTLsetseparator{,}% Set the separator between the columns.

% import data
\DTLloaddb[noheader, keys={thekey,thevalue}]{mydata}{../../database/plane_properties.csv}
% Loads mydata.dat with column headers 'thekey' and 'thevalue'
\newcommand{\PlaneVar}[1]{\DTLfetch{mydata}{thekey}{#1}{thevalue}}




\setlength{\headheight}{14.49998pt}
\addtolength{\topmargin}{-2.49998pt}

\title{„Równowaga podłużna samolotu i siły na sterownicy wysokości”}
\author{Marek Polewski}
\class{Mechanika Lotu 2}
\instructor{dr. inż Maciej Lasek}
\projectName{Projekt 9}
\facultyName{Wydział Mechaniczny Energetyki i Lotnictwa Politechniki Warszawskiej - Zakład Mechaniki}

\renewcommand\figurename{Rys.}%skrocony podpis
\renewcommand\tablename{Tab.}


\begin{document}

\maketitle

\leavevmode\thispagestyle{empty}
\newpage


\tableofcontents

\FloatBarrier
\setcounter{page}{1}


% \section{Samochodzik}

% Zamiast wciągać kresli wole głaskać pieski

% \begin{figure}[h!]
%     \centering
%     \includegraphics[width = 0.4\linewidth]{imgs/burek.jpg}
%     \caption{Burek.jpg}
%     \label{fig:burek}
% \end{figure}
% \FloatBarrier


\section{Wstęp}
Celem projektu jest wyznaczenie kątów wychylenia usterzenia poziomego samolotu koniecznych do zachowania równowangi oraz wyznaczenia siły na drążku.
Obliczenia zostały wykonane w Pythonie. \\
Wysokość przyjęta do obliczeń to $h=3000 m$. Gęstość jej odpowiadająca wynosi $\rho = 0.908 kgm^{-3}$.  \\
Dane z poprzednich projektów:
\begin{itemize}
    \item powierzchnia skrzydeł $S = \PlaneVar{S} m^2$,
    \item powierzchnia usterzenia poziomego $S_h = \PlaneVar{S_h} m^2$,
    \item $\Big(\frac{V_{H\infty}}{V_{\infty}}\Big)^2 = 0.85 $,
\end{itemize}


% TODO FIX citation
Do obliczeń zostały wykorzystane wartości środków ciężkości wyliczone \cite{Proj8}


\begin{table}[h!]
    \centering
    \begin{tabular}{lr} 
    \toprule
                & \multicolumn{1}{l}{$\bar x_c [-] $}  \\ 
    $\bar x_c_1$ & \PlaneVar{bx_c1}                                \\ 
    $\bar x_c_2$ & \PlaneVar{bx_c2}                                \\ 
    $\bar x_c_3$ & \PlaneVar{bx_c3}                                \\ 
    \bottomrule
    \end{tabular}
    \label{tab:srodki}
    \caption{Położenia środka ciężkości samolotu}
\end{table}
\FloatBarrier

\section{Współczynnik momentu podłużnego usterzenia wysokości}
\subsection{Cecha objętościowa usterzenia poziomego}

Wzór na cechę usterzenia:

\begin{equation}
    \kappa'_{h} =  (\bar{x}_{SA_H}-\bar{x}_c)\cdot \frac{S_H}{S} \cdot \Big(\frac{V_{H\infty}}{V_{\infty}}\Big)^2
\end{equation}
, gdzie $\bar{x}_{SA_H} =\PlaneVar{bx_sah}$.  \\
Na podstawie powyższego wzoru uzyskałem:

\begin{table}[h!]
    \centering
    
    \begin{tabular}{rr} 
    \toprule
    \multicolumn{1}{l}{$\bar x_{i}$} & \multicolumn{1}{l}{$\kappa'_{Hi}$}  \\ \hline
    \PlaneVar{bx_c2}                             & \PlaneVar{kappa_c2}                               \\ 
    \PlaneVar{bx_c1}                             & \PlaneVar{kappa_c1}                               \\ 
    \PlaneVar{bx_c3}                             & \PlaneVar{kappa_c3}                               \\
    \bottomrule
    \end{tabular}
    \caption{Cecha objętościowa usterzenia poziomego}
    \label{tab:kappa}
    \end{table}
\FloatBarrier

\subsection{Współczynnik siły nośnej usterzenia poziomego}
Współczynnik siły nośnej usterzenia poziomego wyrażam

\begin{equation*}
    C_{z_H} = a_1 \alpha_H + a_2 \delta_H + a_3 \delta_{H_k}
\end{equation*}


\begin{itemize}
    \item procetowa grubość profilu: $10 \% $,
    \item położenie szczeliny w stosunku do średniej cięciwy płata stabilizatora $0.428\% $,
    \item wydłużenie efektywne usterzenia poziomego: $\Lambda_H = \PlaneVar{Lambda_h}$ 
    \item odwrotność zbierzności usterzenia poziomego: $\frac{1}{\lambda_H} = \PlaneVar{1/lambda_h}$
\end{itemize}

\begin{figure}[h!]
    \centering
    \includegraphics[width = 0.8\linewidth]{imgs/dcz_da_h.jpg}
    \caption{Odczyt charakterystyki $a_1 = \frac{dC_z}{d \alpha}$ dla usterzenia poziomego}
    \label{fig:dczdah}
\end{figure}
\FloatBarrier

Wyznaczanie współcznnika $a_2$

\begin{equation*}
    a_2 = 1.27 \cdot a_1 \cdot \sqrt{\frac{S_{sH}}{S_H}} \Big( 1- 0.2 \cdot \frac{S_{sH}}{S_H} \Big)
\end{equation*}

gdzie $\frac{S_{sh}}{S_H} = \frac{\PlaneVar{Ssh}}{\PlaneVar{Sh}} = \PlaneVar{Ssh/Sh}$ to stosunek powierzchni steru (\ref{fig:Ssh} w \ref{appendix:obl}) do powierzchni usterzenia. 
   
\begin{equation} 
    a_2 = 1.27 \cdot \PlaneVar{a1} \cdot \sqrt{\PlaneVar{Ssh/Sh}}\Big( 1- 0.2 \cdot \PlaneVar{Ssh/Sh} \Big) = \PlaneVar{a2} rad^{-1}
\end{equation}

\section{Kąt zaklinowania usterzenia wysokości}
Obliczenia przyjmę dla prędkości przelotowej równej $V= 106kt = 54.4 m/s$ oraz wysokości $h = 0 m$. Przy tej prędkości płat będzie musiał wytworzyć współczynnik $C_z$ o pewnej wartości. 

\begin{equation}
    C_z = \frac{2mg}{\rho S V^2} = \frac{2 \cdot \PlaneVar{mtow} \cdot 9.81 }{1.225 \cdot \PlaneVar{S} \cdot \PlaneVar{v_cruise}^2} = 0.262
\end{equation}
    
Równanie kąta zaklinowania łopaty:

\begin{equation*}
    \alpha _{zH} = \frac{C_{mbu}}{\kappa'_{H}\cdot a_1}-\frac{C_z}{a}\cdot \Big( 1- \frac{\partial \varepsilon}{\partial \alpha}  \Big)
\end{equation*}

gdzie

\begin{equation}
    \frac{\partial \varepsilon}{\partial \alpha} = \frac{2a}{\pi \Lambda} = \frac{2 \cdot \PlaneVar{a}}{\pi \cdot \PlaneVar{Lambda}} = \PlaneVar{deps_dalpha}
\end{equation}


% \usepackage{tabularray}
\begin{table}[h!]
    \centering
    \begin{tabular}{rrrrr} 
    \toprule
    \multicolumn{1}{l}{$\bar{x}_c$} & \multicolumn{1}{l}{$C_{mbu}$} & \multicolumn{1}{l}{$\kappa'_H$} & \multicolumn{1}{l}{$\alpha_{zH} [rad]$} & \multicolumn{1}{l}{\begin{tabular}[c]{@{}l@{}}$\alpha_{zH} [deg]$\\\end{tabular}}  \\ 
    
    0.12                            & -0.074                      & 0.34                            & -0.10                                & -5.95                                                                           \\ 
    
    0.25                            & -0.039                     & 0.32                        & -0.07                                & -4.16                                                                           \\ 
    
    0.38                            & -0.004                      & 0.31                        & -0.03                                & -2.17                                                                           \\
    \bottomrule
    \end{tabular}
    \caption{Wyznaczanie $\alpha_{zH}$  - kąta zaklinowania płata}
    \label{tab:azh}
    \end{table}
\FloatBarrier

Jako podstawową wartość przyjmuję $\alpha_{zH} \approx -0.007 rad^{-1}= -4.16 deg$.

\section{Kąt wychylenia steru}
Kąt wychylenia steru $\delta_h$ z równania równowagi:

\begin{equation}
    \delta_h = \frac{C_{mbu}}{\kappa'_{H}\cdot a_2}-\frac{a_1}{a_2}\cdot \Bigg[\frac{C_z}{a} \cdot \Big (1- \frac{\partial \varepsilon}{\partial \alpha} \Big)+\alpha_{zH} \Bigg]
\end{equation}


\begin{figure}[h!]
    \centering
    \includegraphics[width = 0.75\linewidth]{imgs/delta_h_cz.jpg}
    \caption{Kąt wychylenia steru niezbędny do zachowania równowagi dla różnego $C_z$}
    \label{fig:deltahcz}
\end{figure}
\FloatBarrier

\begin{figure}[h!]
    \centering
    \includegraphics[width = 0.75\linewidth]{imgs/delta_h_v.jpg}

    \caption{Kąt wychylenia steru niezbędny do zachowania równowagi dla różnego $V$}
    \label{fig:deltahv}
\end{figure}
\FloatBarrier

\begin{table}[h!]
    \centering
    \begin{tabular}{rrrrr}
        \toprule
        $V [ms^{-1}]$ &   $C_z$ &  $\delta_{h1}[^{\circ}]$ &  $\delta_{h2}[^{\circ}]$ &  $\delta_{h3}[^{\circ}]$ \\
        \midrule
        24  &  1.253 &   -12.720 &   -10.166 &    -7.392 \\
        25  &  1.155 &   -11.713 &    -9.158 &    -6.384 \\
        26  &  1.068 &   -10.819 &    -8.265 &    -5.490 \\
        27  &  0.990 &   -10.023 &    -7.468 &    -4.694 \\
        28  &  0.921 &    -9.310 &    -6.756 &    -3.981 \\
        29  &  0.858 &    -8.670 &    -6.116 &    -3.341 \\
        30  &  0.802 &    -8.093 &    -5.539 &    -2.764 \\
        31  &  0.751 &    -7.571 &    -5.016 &    -2.242 \\
        32  &  0.705 &    -7.097 &    -4.542 &    -1.768 \\
        33  &  0.663 &    -6.665 &    -4.111 &    -1.336 \\
        34  &  0.624 &    -6.271 &    -3.717 &    -0.942 \\
        35  &  0.589 &    -5.910 &    -3.356 &    -0.582 \\
        36  &  0.557 &    -5.579 &    -3.025 &    -0.250 \\
        37  &  0.527 &    -5.274 &    -2.720 &     0.054 \\
        38  &  0.500 &    -4.993 &    -2.439 &     0.335 \\
        39  &  0.475 &    -4.734 &    -2.180 &     0.595 \\
        40  &  0.451 &    -4.493 &    -1.939 &     0.835 \\
        41  &  0.429 &    -4.270 &    -1.716 &     1.058 \\
        42  &  0.409 &    -4.063 &    -1.509 &     1.265 \\
        43  &  0.390 &    -3.870 &    -1.316 &     1.458 \\
        44  &  0.373 &    -3.690 &    -1.136 &     1.638 \\
        45  &  0.356 &    -3.522 &    -0.968 &     1.806 \\
        46  &  0.341 &    -3.365 &    -0.811 &     1.964 \\
        47  &  0.327 &    -3.218 &    -0.663 &     2.111 \\
        48  &  0.313 &    -3.079 &    -0.525 &     2.249 \\
        49  &  0.301 &    -2.950 &    -0.395 &     2.379 \\
        50  &  0.289 &    -2.827 &    -0.273 &     2.501 \\
        51  &  0.278 &    -2.712 &    -0.158 &     2.616 \\
        52  &  0.267 &    -2.604 &    -0.050 &     2.724 \\
        53  &  0.257 &    -2.502 &     0.052 &     2.827 \\
        \bottomrule
        \end{tabular}
    \caption{Kąt wychylenia stery niezbędnego do zachowania równowagi}
    \label{tab:deltahv}        
\end{table}
\FloatBarrier


\section{Siła na drążku w warunkach równowagi}
W celu wyznaczenia siły na drążku w warunkach równowagi przyjmuję prędkość $V = \PlaneVar{v_cruise} ms^{-1}$.

\begin{equation}
    P_{dh} = \frac{1}{2}\cdot S_{sH}\cdot c_{sH} \cdot K_{sd} \cdot  V_H^2 \cdot C_{m_{zh}}
\end{equation}
gdzie:
\begin{itemize}
    \item $S_{sH} = \PlaneVar{S_h}$ - pole powierzchni steru wysokości
    \item $c_{sH} = \PlaneVar{Ssh}$- średnia cięciwa steru wysokości
    \item $K_{sd} = \frac{1}{l_d} \cdot \frac{d \delta_h}{d \delta_{dh}} = ???$ - przełożenie liniowe sterownica-ster wysokości
    \item $l_d = 1$ - przełożenie kątowe sterownica-sterk wysokości
    \item $C_{m_{zh}}$ - współczynnik momentu zawiasowego
\end{itemize}

Na podstawie instrukcji \cite{Instrukcja9}, $C_{m_{zh}}$ wyznaczam wzorem:

\begin{equation}
    C_{m_{zh}}=b_0 + b_1*\alpha_H + b_2\delta_H+b_3\delta_{Hk}
\end{equation}


\begin{equation}
    \delta_{Hk} = -\frac{b_1}{b_3}\alpha_H-\frac{b_2}{b_3}\delta_H
\end{equation}

gdzie $\alpha_H$ wyznaczam wzorem:
\begin{equation}
    \alpha_H = \frac{C_z}{a}\cdot \Big( 1-\frac{\partial \varepsilon}{\partial \alpha} \Big)+\alpha_{zH} 
\end{equation}

natomiast współczynnki $b_1$, $b_2$ i $b_3$ na podstawie:

\begin{equation}
    b_1 = b_{1_{\infty}} \cdot \frac{a_1}{a_{1_{\infty}}} = \PlaneVar{b1inf} \cdot \frac{\PlaneVar{a1}}{\PlaneVar{a1inf}} = \PlaneVar{b1}
\end{equation}
\begin{equation}
    b_2 = b_{2_{\infty}} + \Big( \frac{a_1}{a_{1_{\infty}}} -1 \Big) \cdot b_1 \eta_{sH}  = 
    \PlaneVar{b2inf} + \Big(\frac{\PlaneVar{a1}}{\PlaneVar{a1inf}}  -1 \Big) \cdot  \PlaneVar{b1} \cdot \PlaneVar{eta_sh} = \PlaneVar{b2} 
\end{equation}
\begin{equation}
    b_3 = b_{3_{\infty}}K_{kl} \Big( \frac{a_1}{a_{1_{\infty}}} -1 \Big) b_1 \eta_{skH} = 
    \PlaneVar{b3inf}\cdot \PlaneVar{k_kl} \Big( \frac{\PlaneVar{a1}}{\PlaneVar{a1inf}}  -1 \Big) \cdot \PlaneVar{b1} \cdot \PlaneVar{eta_skh} =
    \PlaneVar{b3}
\end{equation}

Wartości współczynników $b_{1_{\infty}},b_{2_{\infty}},b_{3_{\infty}}, \eta_{sH}, \eta_{skH}$ odczytane zostały z tabeli z intrukcji \cite{Instrukcja9}:

\begin{itemize}
    \item $b_{1_{\infty}} = -0.00400$
    \item $b_{2_{\infty}} = -0.00618$
    \item $b_{3_{\infty}}$ na podstawie tabeli 9.2 z \cite{Instrukcja9} oraz wartość $\frac{c{sh}}{c_{h}}= 0.56$	$\frac{c_{kh}}{c_{sh}}=0.345$
    \item $\eta_{sH} = \PlaneVar{eta_sh}$  bo $\frac{S_{sH}}{S_H} = 0.425$
    \item $\eta_{skH} = \PlaneVar{eta_skh}$ bo $\frac{S_{skh}}{S_{sh}} = 0.270$
\end{itemize}
gdzie $\frac{c_{snh}}{c_{sh}} = 0.3$ i  oznacza stosunek średniej cięciwy części steru przed osią obrotu do średniej cięciwy całego steru.


Wartość $K_{kl}$ wycznaczam z równania:

\begin{equation}
    K_{kl} = \frac{S_{sH} \cdot b_{kh}}{S_{kh}\cdot b_{sH}} = \frac{\PlaneVar{Ssh}\cdot \PlaneVar{b_kh}}{\PlaneVar{S0_kh}\cdot \PlaneVar{b_h}} = \PlaneVar{k_kl}
\end{equation}

Teraz na podstawie równania (8) jestem w stanie wyznaczyć współczynniki przy oraz zależność $\delta_{Hk}(\delta_H)$ 
\begin{equation}
    \delta_{Hk} = - \frac{\PlaneVar{b1}}{\PlaneVar{b3}}\alpha_H -\frac{\PlaneVar{b2}}{\PlaneVar{b3}}\delta_H 
\end{equation}

\begin{figure}[h!]
    \centering
    \includegraphics[width = 0.75\linewidth]{imgs/pdh.jpg}

    \caption{Siła na drążku w zależnośći $V$}
    \label{fig:deltahv}
\end{figure}
\FloatBarrier


% \begin{thebibliography}{9}
%     \bibitem{Proj8}
%     Projekt 8 - Moment podłużny samolotu, Marek Polewski październik 2023
% \end{thebibliography}

\appendix
\chapter{Obliczenia powierzchni steru}\label{appendix:obl}
\begin{figure}[h!]
    \centering
    \includegraphics[width = 0.5\linewidth]{imgs/Obliczenia_sh.jpg}
    \caption{Obliczanie pola $S_{sH}$ z rysunku z Projektu 1 \cite{Proj1}}
    \label{fig:Ssh}
\end{figure}
\FloatBarrier



\bibliographystyle{plain} % We choose the "plain" reference style
\bibliography{refs} % Entries are in the refs.bib file

\end{document}