\documentclass[12pt]{sprawozdanie}
\usepackage{color}
\usepackage{tabularray}
\usepackage{tikz}
\usepackage{polski}
\usepackage{graphicx} % Required for inserting images
\usepackage{multicol}
\usepackage{booktabs}

\usepackage{lipsum}
\usepackage[T1]{fontenc}
% package to open file containing variables
\usepackage{datatool, filecontents}
\DTLsetseparator{,}% Set the separator between the columns.

% import data
\DTLloaddb[noheader, keys={thekey,thevalue}]{mydata}{../../database/plane_properties.csv}
% Loads mydata.dat with column headers 'thekey' and 'thevalue'
\newcommand{\PlaneVar}[1]{\DTLfetch{mydata}{thekey}{#1}{thevalue}}




\setlength{\headheight}{14.49998pt}
\addtolength{\topmargin}{-2.49998pt}

\title{„Równowaga podłużna samolotu i siły na sterownicy wysokości”}
\author{Marek Polewski}
\class{Mechanika Lotu 2}
\instructor{dr. inż Maciej Lasek}
\projectName{Projekt 9}
\facultyName{Wydział Mechaniczny Energetyki i Lotnictwa Politechniki Warszawskiej - Zakład Mechaniki}

\renewcommand\figurename{Rys.}%skrocony podpis
\renewcommand\tablename{Tab.}


\begin{document}

\maketitle

\leavevmode\thispagestyle{empty}
\newpage


\tableofcontents

\FloatBarrier
\setcounter{page}{1}


% \section{Samochodzik}

% Zamiast wciągać kresli wole głaskać pieski

% \begin{figure}[h!]
%     \centering
%     \includegraphics[width = 0.4\linewidth]{imgs/burek.jpg}
%     \caption{Burek.jpg}
%     \label{fig:burek}
% \end{figure}
% \FloatBarrier


\section{Wstęp}
Celem projektu jest wyznaczenie kątów wychylenia usterzenia poziomego samolotu koniecznych do zachowania równowangi oraz wyznaczenia siły na drążku.
Obliczenia zostały wykonane w Pythonie. \\
Wysokość przyjęta do obliczeń to $h=3000 m$. Gęstość jej odpowiadająca wynosi $\rho = 0.908 kgm^{-3}$.  \\
Dane z poprzednich projektów:
\begin{itemize}
    \item powierzchnia skrzydeł $S = \PlaneVar{S} m^2$,
    \item powierzchnia usterzenia poziomego $S_h = \PlaneVar{S_h} m^2$,
    \item $\Big(\frac{V_{H\infty}}{V_{\infty}}\Big)^2 = 0.85 $,
\end{itemize}

Do obliczeń zostały wykorzystane wartości środków ciężkości wyliczone \cite{Proj8}



\begin{table}[h!]
    \centering
    \begin{tabular}{lr} 
    \toprule
                & \multicolumn{1}{l}{$\bar x_c [-] $}  \\ \hline
    $\bar x_c_1$ & \PlaneVar{bx_c1}                                \\ \hline
    $\bar x_c_2$ & \PlaneVar{bx_c2}                                \\ \hline
    $\bar x_c_3$ & \PlaneVar{bx_c3}                                \\ 
    \bottomrule
    \end{tabular}
    \label{tab:srodki}
    \caption{Położenia środka ciężkości samolotu}
\end{table}
\FloatBarrier

\section{Współczynnik momentu podłużnego usterzenia wysokości}
\subsection{Cecha objętościowa usterzenia poziomego}

Wzór na cechę usterzenia:

\begin{equation}
    \kappa'_{h} =  (\bar{x}_{SA_H}-\bar{x}_c)\cdot \frac{S_H}{S} \cdot \Big(\frac{V_{H\infty}}{V_{\infty}}\Big)^2
\end{equation}
, gdzie $\bar{x}_{SA_H} =\PlaneVar{bx_sah}$.  
Na podstawie powyższego wzoru uzyskałem:

\begin{table}[h!]
    \centering
    
    \begin{tabular}{rr} 
    \toprule
    \multicolumn{1}{l}{$\bar x_{i}$} & \multicolumn{1}{l}{$\kappa'_{Hi}$}  \\ \hline
    \PlaneVar{bx_c1}                             & \PlaneVar{kappa_c1}                               \\ \hline
    \PlaneVar{bx_c2}                             & \PlaneVar{kappa_c2}                               \\ \hline
    \PlaneVar{bx_c3}                             & \PlaneVar{kappa_c3}                               \\
    \bottomrule
    \end{tabular}
    \caption{Cecha objętościowa usterzenia poziomego}
    \label{tab:kappa}
    \end{table}
\FloatBarrier

\subsection{Współczynnik siły nośnej usterzenia poziomego}
Współczynnik siły nośnej usterzenia poziomego wyrażam

\begin{equation*}
    C_z_H = a_1 \alpaha_H + a_2 \delta_H + a_3 \delta_H_k
\end{equation*}

\begin{itemize}
    \item procetowa grubość profilu $10 \% $,
    \item położenie szczeliny w stosunku do średniej cięciwy płata stabilizatora $0.428$,
    \item wydłużenie efektywne usterzenia poziomego 
    \item odwrotność zbierzności usterzenia poziomego
\end{itemize}



% \begin{thebibliography}{9}
%     \bibitem{Proj8}
%     Projekt 8 - Moment podłużny samolotu, Marek Polewski październik 2023
% \end{thebibliography}


\bibliographystyle{plain} % We choose the "plain" reference style
\bibliography{refs} % Entries are in the refs.bib file

\end{document}